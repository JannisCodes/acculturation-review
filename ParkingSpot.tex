\documentclass[nobib]{tufte-handout}
%\geometry{showframe} % for debugging purposes -- displays the margins
\usepackage{amsmath}

% Set up the images/graphics package
\usepackage{graphicx}
\setkeys{Gin}{width=\linewidth,totalheight=\textheight,keepaspectratio}
\graphicspath{{graphics/}}

\title[Psychological Acculturation - Notes]{Psychological Acculturation: \\
Parking spot for removed sections}
\author[Kreienkamp et al.]{}
%\date{March 28, 2021}  % if the \date{} command is left out, the current date will be used

% The following package makes prettier tables.  We're all about the bling!
\usepackage{booktabs}

% The units package provides nice, non-stacked fractions and better spacing for units.
\usepackage{units}

% The fancyvrb package lets us customize the formatting of verbatim environments.  We use a slightly smaller font.
\usepackage{fancyvrb}
\fvset{fontsize=\normalsize}

% Small sections of multiple columns
\usepackage{multicol}

% Provides paragraphs of dummy text
\usepackage{lipsum}

% strike out text
\usepackage[normalem]{ulem}

% highlight text
\usepackage{soul}

% APA citations
\bibliographystyle{plain}
\usepackage[style=apa, sortcites=true, sorting=nyt, backend=biber, natbib=true, uniquename=false, uniquelist=false, useprefix=true]{biblatex}
% add reference library file
\addbibresource{references.bib}

% These commands are used to pretty-print LaTeX commands
\newcommand{\doccmd}[1]{\texttt{\textbackslash#1}}% command name -- adds backslash automatically
\newcommand{\docopt}[1]{\ensuremath{\langle}\textrm{\textit{#1}}\ensuremath{\rangle}}% optional command argument
\newcommand{\docarg}[1]{\textrm{\textit{#1}}}% (required) command argument
\newenvironment{docspec}{\begin{quote}\noindent}{\end{quote}}% command specification environment
\newcommand{\docenv}[1]{\textsf{#1}}% environment name
\newcommand{\docpkg}[1]{\texttt{#1}}% package name
\newcommand{\doccls}[1]{\texttt{#1}}% document class name
\newcommand{\docclsopt}[1]{\texttt{#1}}% document class option name

% make question with red triangle
\newcommand\Question[1][2ex]{%
  \renewcommand\stacktype{L}%
  \scaleto{\stackon[1.3pt]{\color{red}$\triangle$}{\tiny\bfseries ?}}{#1}}%
  
% add definition sections
\newtheorem{definition}{Definition}

% add quote section
\usepackage{csquotes}

% framed box section
\usepackage{framed}
\emergencystretch=1em

\begin{document}

\maketitle % this prints the handout title, author, and date

\section{Parking Spot}
Don't know where this goes:

\paragraph{Inter-connectedness Focus Group}
It remains important to point out that these themes emerged as part of the focus coding but were not discussed in isolation from one another. The group, for example, pointed out that most of the cognitive and affective acculturation aspects are often facilitated through (structured) social behavioral activities such as school, work, sports, music, and associations. Similarly, the feelings of becoming part of society (including descriptions of feeling useful, self-esteem, and purpose) were related to behavioral adaptation (contact, and work), and cognitive adaptation (developing a complex identity narrative beyond the refugee identity). Moreover, the group also highlighted during their discussions, that in practice acculturation was mostly a temporal process rather than a static end goal. While the group did discuss possible comparisons of how one person might be more acculturated than another, an overwhelming majority of the discussion centered around the acculturation process and aspects that are elementary to any migration experience. 

\paragraph{Focus Group conclusion}
In sum, the voices of experts in the field and of those actively going through the migration process highlighted the importance of focusing on the common elements of individual experiences in order to comprehensively understand the elements of the concept. In our coding of the discussion we identified affect, behavior, cognition, and desire related conceptualizations of acculturation experience.

\paragraph{Applications}
% Applications
% relate back to problem: Theory and Application; Past and Future
And in its application the different aspects of the acculturation experience should apply both on a theory level (i.e., what we mean with psychological acculturation and how it relates to other concepts) as well as an empirical measurement level (i.e., how we capture and assess the concept). In this application the structure of the experience framework relate back to the current hurdles within the field.

Looking at the past literature, the ABCD structure allows us to asses and compare past theories, models, and measurements of the concept. Which aspects of the acculturation experience have been considered as part of the concept in theories, validated measures, empirical investigations, and interventions? And if multiple aspects have been considered, what are patterns of co-occurrences, how were the aspects related to one another, and how have they related to other concepts?

Looking at the status quo and future directions, the ABCD aspects allow us to make focus and selection decisions for theories, models, and measurements. Which aspect is relevant for a particular theory or measurement, and how do these elements relate to one another, as well as other concepts? And how do these elements develop over time in their relations with themselves and other aspects and concepts?

\paragraph{Focus}
% focus
One important note on our focus is the differentiation between dominant and non-dominant culture members. While psychological acculturation can apply to any inter-cultural contact \citep[including, for example, the contact between indigenous people and colonial settlers, e.g.,][]{Berry1974}, a majority of theories and studies have focused on cross-national immigrants. And while the framework we propose is applicable to psychological acculturation in its wider sense, its development and our focus in the following sections will lay mainly with the experience of non-dominant immigrants (as is common in the field).

\paragraph{Inter-connectedness General}
%% Host-Migrant Interconnectedness:
% Focus on migrant perspective because often disadvantaged position but any interdependent aspect should show up in ABCD + framework can also be used for majority perspective (but not focus of this paper)
Before we move on to the systematic review and our application of the conceptual framework, we would like to remark on our focus on the migrant's rather than the dominant cultural group member's experiences. 
While this framework explicitly focuses on the migrant's experience as the structuring experience, this is not meant to put the sole responsibility of ``migration success'' on the newcomer or ignore the interconnectedness of intercultural exchange. We focus on the migrant experience as the structuring element because migrants usually find themselves in disadvantaged positions. 

We would also like to address the issue of interactive acculturation. \citet{Bourhis1997a} and others, have often highlighted that cultural adaptation is essentially interactive and one should not consider the acculturation process of one individual or group without the consideration of the other group and the interaction partners. We agree, and while we address some of these issues explicitly in the framework, we would also like to highlight that within the experience framework aspects of cultural adaptation that are interconnected, co-dependent, or depend on the dominant group members likely influence the individual's migration experience. As an example, if a migrant is frequently confronted with exclusion and hostility, this will likely affect their motivations, emotions, thoughts, and behaviors. 
We would also like to highlight that the framework is equally applicable to the cultural adaptation experience of dominant cultural group members in their interactions with or perceptions of migrants. Although not the focus of this paper, the same assessments of measures, definitions, and understandings should be able to be conducted with the experience of the dominant group.
Additionally, this framework is not meant as an exhaustive or exclusive list of all adaptation aspects but rather offers a structural framework to analyze and understand the broad experience of cultural adaptation.
%Migration is a complex trans-formative experience within different areas of personal- and social life. 

\section{Notes to self}
\begin{itemize}
  \item assess domains based on framework we propose
  \item life domains (in part) based on Humanitas
\end{itemize}


\newpage
\printbibliography

\end{document}