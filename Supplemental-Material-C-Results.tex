\subsection{Methodological Literature}

To gain a general understanding of contextual factors within the
validated studies, we also assessed cross-study patterns of cultural,
individual, situational, and process-related focus points.

\paragraph{Country}

To assess the cultural contexts for which scales were validated, we
assessed the migrants' countries of settlement as well as the countries
of origin. We found that most scales investigated a single host country
(\textit{N} = 204) and most investigated one country of origin
(\textit{N} = 140). There were only 29 scales that were validated for
more than one receiving country. Looking at the country patterns, we
found that an overwhelming number of scales were validated within a U.S.
American settlement context (\textit{N} = 126). But also the remaining
receiving societies were mostly `western' countries (e.g., Canada, The
Netherlands, The United Kingdom, Israel, Australia) with non-western
settlement contexts, such as Taiwan, Nepal, or Russia, not being
investigated across more than two study. For the migrant origin
societies there was slightly more variation. There were a few migrant
groups that were investigated specifically (e.g., Mexico: 25, China:13,
South Korea: 12), however most validation studies targeted broader
categories of migrants (any migrants: 53, Asian: 10, Hispanic: 10,
LatinX: 12). This also made it difficult to identify patterns of
cultural combinations investigated (apart from Mexican and LatinX
migrants in the United States).

\paragraph{Sample}

To assess the role different groups of individuals targeted in the scale
validations, we coded the types of samples recruited for the validation
studies. A majority of studies sampled any consenting adult from the
migrant group of interest (\textit{N} = 126). As seems common in
academic research, a larger portion of the validated scales relied on
young migrants or students (\textit{N} = 66). Interestingly, only small
minority of validated scales targeted more vulnerable groups, such as
clinical samples (\textit{N} = 3) or refugees (\textit{N} = 6) --
despite a considerable focus on these groups within the broader
literature. Given the small cell counts, we did not investigate
differences in the experience measures across the different samples.

\paragraph{Domains}

To assess the situational focus within the validated scales, we assessed
the number of domains within each scale as well as more common domains
across the scales. The scales included an average of 4.23 life domains
(\textit{SD} = 2.71). The most common domains to be included were
`friends/aquaintances' (\textit{N} = 155, 66.52\%), `home/family'
(\textit{N} = 145, 62.23\%), and `entertainment/media/news' (\textit{N}
= 105, 45.06\%). Looking at combinations of domains that were commonly
assessed together, a number of patterns emerged within the bi-variate
relationships. One cluster was, for example around the common domain
`friends/aquaintances', which had a high proportion of co-occurences
with `entertainment/media/news' (\textit{r} = 0.46, \textit{p}
\textless{} .001) and with `recreation/sport/art' (\textit{r} = 0.39,
\textit{p} \textless{} .001). However, when look at unique combinations
of domains we obsered an essentially scattered field. Within the 233
scales we coded, we found a total of 138 different domain combinations.
A considerable proportion of scales focused on a unique combination of
life domains (44.21\%) and a large majority of domain combinations was
used by less then five percent of the scales (85.41\%; also see
Supplemental Material B). Thus, while there was large variation between
the scales in the number, and diversity of domains, the most frequently
mentioned domains were in line with the life domains proposed in the
literature \citep[e.g.,][]{Arends-Toth2007}. Yet again, given the large
variability between studies, we did not investigate differences in
experience elements across the different situational domains.

\subsection{Empirical Literature}

To gain a general understanding of contextual factors within the broader
empirical studies, we again assessed cross-study patterns of cultural,
individual, situational, and process-related focus points.

\paragraph{Country}

To assess the cultural contexts on which the authors focused, we again
assessed the migrants' countries of settlement as well as the countries
of origin. Similar to the validations, an overwhelming number of scales
were validated within a North American settlement context (United
States: \textit{N} = 280, Canada: \textit{N} = 44). But also the
remaining receiving societies were mostly `western' -- Western Europe
(e.g., The Netherlands, United Kingdom, Germany, Italy, Spain),
Australasia (Australia, New Zealand), Russia, and Israel. And only 25
studies focused on data from multiple receiving societies.

When it came to the migrants' country of origin, a majority of studies
were indifferent to migrants background and simply recruited any
consenting migrant (\textit{N} = 108), or recruited a category of
migrants (e.g., LatinX or Hispanic: \textit{N} = 67, Asian:\textit{N} =
26 African: \textit{N} = 14). Among the individual countries target
there was a particular focus on the east and south-east Asian region
(e.g., China: \textit{N} = 48, South Korea: \textit{N} = 37, Vietnam:
\textit{N} = 22). Yet, different from the scale validations, there was a
large variety of different origin countries that were included in less
than five studies (\textit{N} = 103 regions were targeted less than five
times). Thus, the receiving countries mainly mirrored those for which
scales were validated, yet there was an extensive number origin
countries which were investigated individually or migrants were
considered irrespective of their cultural origin. Moreover, it was again
not possible to identify distinct cultural clusters within the
literature that would be large enough to compare measures of cultural
adaptation.

\paragraph{Sample}

To assess the role different groups of individuals targeted in the
empirical work, we again coded the types of samples recruited for the
studies. A majority of studies sampled any consenting adult from the
migrant group of interest (\textit{N} = 282, 53.61\%). Of the targeted
sampling strategies, most recruited young migrants (\textit{N} = 97,
18.44\%) women (\textit{N} = 50, 9.51\%), or refugees (\textit{N} = 35,
6.65\%). The remaining fifth of the studies recruited other more
specific samples (e.g., nurses, athletes, Muslims). Interestingly, even
though a large portion of papers focused on mental health outcomes, only
7 studies (1.33\%) recruited clinical migrant samples. These results
speak to the case that either the sub-populations are too small to be
sampled properly or relatively few empirical studies actually take
individual differences into account in their sample selection. Studies
may still address individual differences within the data description and
within the modeling approaches (e.g., controlling for gender), yet it
seems that inter-sectional idiosyncrasies did not seem to play a major
role in the targeting of samples. Moreover, cell counts were again
unbalanced and we did not assess experience differences between the
samples.

\paragraph{Domains}

To capture the situational focus of the authors, we coded which life
domains the utilized measures referred to. We coded which life domains
the authors referred to, either as part of subscale labels, factor
labels, explicit commentary of the authors, or clear question wordings
to gain an understanding of the situational focus the authors chose.
However, we did not code the theoretical situational life domains
because such an undertaking would be beyond the scope of this paper. And
the conceptual utility of such a coding was already explored for the
methodological literature.
