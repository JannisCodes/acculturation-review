\begin{table}%[hbt]
\caption{Synthesis Summary and Future Perspectives.}
\label{tab:SummaryTbl} 

\footnotesize

\begin{tabular}{>{\raggedright\arraybackslash}p{0.50\linewidth} 
>{\raggedright\arraybackslash}p{0.50\linewidth}}

\hline 
Critical issues identified in systematic review &
Perspective / Suggestions \\ 
\hline

\vspace{-0.5em} \hangindent=0.55cm 1.~ Theoretical and empirical conceptualizations of psychological acculturation have been diverse and unstructured.  & 
\vspace{-0.5em} The affect, behavior, cognition, desire distinctions could be used to structure acculturation conceptualizations. \\ 

\vspace{-0.5em} \hangindent=0.55cm 2.~ Empirical studies focus on cross-sectional outcome conceptualizations while theories predominantly conceptualize psychological acculturation as a process. & 
\vspace{-0.5em} In empirical works a stronger focus on longitudinal acculturation assessments is needed to congruently test theories. \\ 

\vspace{-0.5em} \hangindent=0.55cm 3.~ Theories include substantially more experience aspects in their conceptualization than empirical studies. & 
\vspace{-0.5em} Empirically, investigations of more acculturation aspects are needed to congruently test theories. \\ 

\vspace{-0.5em} \hangindent=0.55cm 4.~ There has been little empirical focus on emotional and motivational aspects, even though they are important in theories and qualitative discussions. & 
\vspace{-0.5em} To close this gap, empirical studies that investigate affect and desire are needed. \\ 

\vspace{-0.5em} \hangindent=0.55cm 5.~ Theories have been investigated within individual experience aspects (e.g., behavioral or cognitive orientations), but effects have rarely been compared across aspects.  & 
\vspace{-0.5em} There is a need to compare the relationship of different experience aspects with other concepts. E.g., does behavioral acculturation have the same impact on health as emotional acculturation? \\ 

\vspace{-0.5em} \hangindent=0.55cm 6.~ In theoretical and empirical work, experience aspects are commonly considered independently. & 
\vspace{-0.5em} There is a need to investigate the relationships between different experience aspects. \\ 

\vspace{-0.5em} \hangindent=0.55cm 7.~ Psychological and cultural adaptation (as a form of acculturation) have often been conceptualized inconsistently. & 
\vspace{-0.5em} Future investigations and interventions could consider functionality and adaptation within each experience aspect. \\ 

\vspace{-0.5em} \hangindent=0.55cm 8. We identified \nTheo\ (mostly independent) theoretical works. & 
\vspace{-0.5em} Future research should assess the possibility of theoretical synthesis \citep[e.g.,][]{Maertz2016}. The experience framework might offer a conceptual lens for such a synthesis.\\ 

\vspace{-0.5em} \hangindent=0.55cm 9. The normative aim of acculturation conceptualizations are often unclear (e.g., does the conceptualization aim to benefit an individual or society?). & 
\vspace{-0.5em} There is a need to discuss the normative expectations of acculturation conceptualizations within empirical and theoretical work \citep[e.g.,][]{Ager2008a}. \\ 

\vspace{-0.5em} \hangindent=0.65cm 10.~ The choice of investigated acculturation aspects have often remained elusive in methodological and applied empirical literature. & 
\vspace{-0.5em} For replications, comparisons, and theoretical synthesis, research and intervention choices need to be transparent. Which aspect is focused on? Why is an aspect (ir)relevant to the project? \\ 

\vspace{-0.5em} \hangindent=0.65cm 11.~ Operationalizations and measurements of acculturation are often reported unclearly (especially with ad-hoc measures or non-validated modifications and non-disclosed items). & 
\vspace{-0.5em} As long as the field faces conceptual issues, transparency in measurement remains important. Either items or clear content descriptions should be available.\\ 

\vspace{-0.5em} \hangindent=0.65cm 12. The non-dominant target group has often been defined very broadly (e.g., any migrant, Asia, Spanish-speaking, third world). & 
\vspace{-0.5em} Research questions, conceptualizations, and measurements concerning acculturation should be specific to all considered cultural contexts or should be transferable across all considered cultural contexts. \\ 

\vspace{-0.5em} \hangindent=0.65cm 13. Acculturation measures are often validated within specific cultural contexts but are applied within other cultural contexts. & 
\vspace{-0.5em} Future research needs to assess the impact of non-validated scales. \\ 

\vspace{-0.5em} \hangindent=0.65cm 14. Empirical work has had a strong focus on clinical outcomes but utilized few clinical samples. & 
\vspace{-0.5em} Differences between clinical and non-clinical samples should be assessed where researchers focus on clinical outcomes. \\ 

\hline

\end{tabular}
\end{table}
