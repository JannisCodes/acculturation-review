% define document type (i.e., template. Here: A4 APA manuscript with 12pt font)
\documentclass[man, 12pt, a4paper]{apa7}

% change margins (e.g., for margin comments):
%\usepackage{geometry}
% \geometry{
% a4paper,
% marginparwidth=30mm,
% right=50mm,
%}

% add packages
\usepackage[american]{babel}
\usepackage[utf8]{inputenc}
\usepackage{csquotes}
\usepackage{hyperref}
\usepackage[style=apa, sortcites=true, sorting=nyt, backend=biber, natbib=true, uniquename=false, uniquelist=false, useprefix=true]{biblatex}
\usepackage{authblk}
\usepackage{graphicx}
\usepackage{setspace,caption}
\usepackage{subcaption}
\usepackage{enumitem}
\usepackage{lipsum}
\usepackage{soul}
\usepackage{xcolor}
\usepackage{fourier}
\usepackage{stackengine}
\usepackage{scalerel}
\usepackage{fontawesome}
\usepackage[normalem]{ulem}
\usepackage{longtable}
\usepackage{amsmath}
\usepackage{afterpage}
\usepackage{float}
\usepackage{array}

% add definition sections
\newtheorem{definition}{Definition}

% Select what to do with to-do notes: 
% \usepackage[disable]{todonotes} % notes not showed
% \usepackage[draft]{todonotes}   % notes showed

% make warning with red triangle
\newcommand\Warning[1][2ex]{%
  \renewcommand\stacktype{L}%
  \scaleto{\stackon[1.3pt]{\color{red}$\triangle$}{\tiny\bfseries !}}{#1}}%

% make question with red triangle
\newcommand\Question[1][2ex]{%
  \renewcommand\stacktype{L}%
  \scaleto{\stackon[1.3pt]{\color{red}$\triangle$}{\tiny\bfseries ?}}{#1}}%

% formatting links in the PDF file
\hypersetup{
pdfpagemode={UseOutlines},
bookmarksopen=true,
bookmarksopenlevel=0,
hypertexnames=false,
colorlinks   = true, %Colours links instead of ugly boxes
urlcolor     = blue, %Colour for external hyperlinks
linkcolor    = blue, %Colour of internal links
citecolor   = cyan, %Colour of citations
pdfstartview={FitV},
unicode,
breaklinks=true,
}

% language settings
\DeclareLanguageMapping{american}{american-apa}

% add reference library file
\addbibresource{references.bib}

% Title and header
\title{The Migration Experience: A Conceptual Framework and Systematic Review of Psychological Acculturation}
\shorttitle{Acculturation Experience Framework}

% Authors
\author[*,1,2]{Jannis Kreienkamp}
\author[1,2]{Kai Epstude}
\author[1,2]{Laura F. Bringmann}
\author[1,2]{Peter de Jonge}
\affiliation{\hfill}
\affil[1]{University of Groningen, Department of Psychology}
\affil[2]{Author order and additional authors still to be discussed (currently sorted by first name)}

\authornote{
   \addORCIDlink{* Jannis Kreienkamp}{0000-0002-1831-5604}

We have no known conflict of interest to declare.

Correspondence concerning this article should be addressed to Jannis Kreienkamp, Department of Psychology, University of Groningen, Grote Kruisstraat 2/1, 9712 TS Groningen (The Netherlands).  E-mail: j.kreienkamp@rug.nl}

\leftheader{Kreienkamp}

% Abstract
\abstract{
\Warning\ Needs to be updated!\\One of the key challenges to researching psychological acculturation is an immense heterogeneity in theories and measures. These inconsistencies make it difficult to compare past literature on acculturation, hinder straight-forward measurement selections, and hampers the development of an overarching framework. To structure our understanding of the migration process, we propose to utilize the four basic elements of human experiences (motivations, emotions, thoughts, and behaviors) as a conceptual framework. We use this framework to build a theory-driven literature synthesis and find that the past methodological and empirical literature as well as theoretical models have understudied the more internal aspects of acculturation (motivations and emotions) and have often fallen short of capturing all four aspects of the migration experience. 

[currently 117 words]
}

\keywords{Keyword \#1, Keyword \#2, ...}

% set indentation size
\setlength\parindent{1.27cm}

% Start of the main document:
\begin{document}

% add title information (incl. title page and abstract)
\maketitle

% **CHEAT SHEET / LEGEND**
%
% Comments:
% '%' starts a comment in LaTeX (not printed)
% '\todo[inline]{} makes orange boxes in PDF
% '\marginpar{}' notes in margins
% '\footnote{}' footnote
% '\Warning' important note indicator in PDF (triangle with exclamation mark)
% '\Question' question note indicator in PDF (triangle with question mark)
%
% Citation (with Natbib citation style):
% '\citep[e.g.][p. 15]{CitationKey}' citation in parentheses "(e.g., Berry, 2003, p. 15)"
% '\citet{CitationKey}' citation in text "Berry (2003)"
% '\citealt' and '\citealp' alternate citation without parentheses
% '\citeauthor' and '\citeyear' only year or author
% 
% Headings:
% '\part{}' and '\chapter{}' only relevant for multi-part or multi-chapter documents
% '\section{}' heading level 1
% '\subsection{}' heading level 2
% '\subsubsection{}' heading level 3
% '\paragraph{}' heading level 4
% '\subparagraph{}' heading level 5
%
% formatting:
% '\textbf{}' text bold font
% '\textit{}' text italic font
% '\underline{}' text underline
% '\sout{}' text strike out
% '\textsc{}' text small caps
% '\vspace{1em}' add vertical space
% '\hspace{1em}' add horizontal space
% '\\' new line (i.e., line break)
% '\pagebreak' start new page (i.e., page break)
% '\noindent' do not indent current line (e.g., current paragraph)
% 'begin{center}...end{center}' center text or object
%
% Math mode:
% '$\alpha = .8$' mathematical equation inline
% '$$\hat{y} = b_0 + b_1x$$' mathematical equation in its own line
% '\begin{equation}...\end{equation}' multi-line equation
% '\approx' approximate symbol
% '\neq' not equal
% '\bar' mean bar over letter
% '\pm' plus minus sign 
% '^{}' superscript
% '_{}' subscript
% '\fraq{numerator}{denominator}' fraction
% '\sqrt[n]{}' square root
% '\sum_{k=1}^n' sum for 1 through n
%
% Insert things from elsewhere:
% '\input{filename}' inputs the raw (tex) file as a command (e.g., tables and R-Markdown imports)
% '\include{filename}' includes section on new page (incl. possible auxiliary info)
% '\includegraphics[settings]{filename}' add a figure or graph
% '\caption{}' adds a caption to a table or figure
% '\label{}' labels sections, tables, figures, etc. so that they can be referred to.
% '\ref{}' refer to a labelled sections, tables, figures, etc.
% '\begin{enumerate}...\end{enumerate}' numbered list
% '\begin{itemize}...\end{itemize}' bullet-ed list
% '\item' item in list section 
%
% Symbols:
% '\&' and sign
% '\%' percent sign
% '\_' three dotes
% '\#' hash symbol
% ------------------------------------------------------------------

\section{Introduction [not an actual header - label for our discussion]}

% Relevance
% Long discussed and relevant to society
The question of how people change when they get into continuous first-hand contact with other cultures is probably as old as the history of human migration and remains an important issue for many societies around the world. 
Almost 4,000 years ago Ipuwer of Egypt wrote "those who were once Egyptians [have become] aliens" (\citeauthor{Ipuwer2003}, ca. 1250 B.C.E./2003, Admonitions of Ipuwer 4:1) because of their contact with foreigners. Nearly 2,000 years later Plato similarly bemoaned the "blending of characters" (\citeauthor{Plato1926}, ca. 384 B.C.E./1926, Laws 12:949e ff.) when the good people of Athens got into contact with other cultures. Another 2,000 years later, over the past 80 years, researchers of the psychological sciences have proposed hundreds of models and measurements for this phenomenon of "psychological acculturation" \citep[][]{Rudmin2003a}. Yet, despite enormous theoretical and applied advances, it remains unclear what psychological acculturation exactly entails. A conceptual framework allowing for a synthesis of the past literature on psychological acculturation is still missing \citep{Birman2014c}.

% Problem - Definitions
% definitions vague and unstructured
The most widely reported definition of 'acculturation', as it is understood within the psychological tradition, is probably the definition by the Social Science Research Council \citep[][]{Redfield1936}:
\begin{quote}
    \begin{definition}[Acculturation]
        Acculturation comprehends those phenomena which result when groups of individuals having different cultures come into continuous first-hand contact, with subsequent changes in the original culture patterns of either or both groups. (p. 149)
    \end{definition}  
\end{quote}
Following this definition, acculturation is any observable event (i.e., phenomenon) or change that is due to a contact with a another culture. Notably, the terms of “phenomena” and “changes” remain decidedly vague and define acculturation as both a process and an outcome at the same time. Moreover, the phrase "groups of individuals" suggests that acculturation occurs as two related, yet separate, processes of cultural and psychological acculturation \citep[also see,][]{Sam2006b, Berry2005}.

In this review we will focus on the individual level and assess aspects of 'psychological acculturation', as that is the focus of most psychological research. For this more micro level concept many have used the definition proposed by \citet[][]{Sam2006b}, based on a research paper by \citet{Graves1967}: 
\begin{quote}
    \begin{definition}[Psychological Acculturation]
        “Psychological acculturation” refers to the changes an individual experiences as a result of being in contact with other cultures, or participating in the acculturation that one’s cultural or ethnic group is undergoing (Graves, 1967). (p. 14)
    \end{definition}
\end{quote}
Here the main focus of the definition lies on the individual experience of changes, which also remains relatively broad and gives little structural guidance. Unsurprisingly then, later reviews have shown that many different study targets fall within the broad definitions and past literature has struggled to structurally bring these different perspectives together.

% Problem - Past Conceptual Literature
% All the different aspects are not structured
As the literature on psychological acculturation has grown exponentially over past decades \citep{Rudmin2003a}, several reviews \citep[e.g.,][]{DeLeersnyder2017, Matsudaira2006, Celenk2011}, conceptual reflection papers \citep[e.g.,][]{Ward2012, Berry1997b}, and even a handbook have been published \citep{Sam2006a}. And while these reviews have documented immense progress in the study of psychological acculturation, have collected many of the researched topics, and have offered many useful theories to address important issues, the broader heterogeneity of all of these conceptual elements remains a key issue.

We find an illustration in the applications of prominent structural theories. One such, extremely valuable theory is that of acculturation orientations (based on Berry and colleagues (\citeyear{Berry1997b, Berry2003}) acculturation attitudes). These orientations put forwards the idea that psychological acculturation includes a two-dimensional change process in peoples' preferences of their heritage culture and the new culture \citep[e.g.,][]{Ward2001, Berry2003}. However, what these preferences actually pertain to has remained unstructured and so in the past these orientations have focused among other things on: attitudes, attachments, goals, identifications, as well as the choice and use of cultural elements, such as language, foods, or dresses \citep[e.g.,][]{Rudmin2003a}.

Similarly challenges can be seen with the proposal that acculturation orientations are distinct from more long-term adaptations processes \citep{Searle1990, Ward2001, Berry2003}. The scope and structure of these psychological and socio-cultural adaptation processes have remained similarly broad and have for example included aspects such as satisfaction and well-being, as well as cultural skills and performance. 

Adding to the conceptual difficulties is the case that in empirical practice orientation and adaptation distinctions are less commonly observed (for exceptions see \citealp{ICSEYteam2006, Berry2006b, TeLindert2008a}). Instead, researchers more commonly focus on acculturation in terms of different 'acculturation aspects' \citep{Arends-Toth2006a}, which might include any of the orientation, identity, or adaptation elements. In sum, it thus remains unclear what aspects the concept exactly entails, how these aspects are organized, and measurements are inconsistent across studies and interventions.

% Problem - Consequences
% Theory and Application; Past and Future
This heterogeneity of aspects presents fundamental challenges to researchers, practitioners, and policy-makers in the field. In particular, the absence of a conceptual framework causes issues in the synthesis of past literature and the development of new theories and measurements.
Looking back at past theories and measures, the diversities of included or excluded constructs makes it virtually impossible to compare different effects and outcomes, which makes it difficult to integrate them without a framework that comprehensively organizes the different aspects \citep{Taft1981}.
Looking forward, it in turn remains difficult to select past elements and develop new theories and measurements. A conceptual framework would be necessary to make informed and transparent choices on which aspects are (ir)relevant to a given research question and how they relate to one another.
Given these challenges, some have even suggested that psychological acculturation should not be measured until common terminologies and frameworks are available \citep{Escobar2000}.

% Aim - General
% Therefore we suggest a conceptual framework
The aim of this paper is, therefore, to offer a descriptive conceptual framework to analyze, measure, and understand the concept of psychological acculturation. Such a framework has a different objective than previous efforts to catalogue literature on acculturation \citep[e.g.,][]{Castels2003}, build multidimensional measures of integration \citep[e.g.,][]{Harder2018}, normative frameworks \citep[e.g.,][]{Ager2008a}, or theories of acculturation \citep[e.g.,][]{Berry2005}. Rather than offering a new measurement, definition, or theory, we aim to build a framework to assess and compare any of these conceptual elements. 

% Aim - Specific
% We do this by using ABCD
To build a framework that would comprehensively structure the concept of psychological acculturation across a wide range of contexts, we propose to utilize the basic elements of human experiences and motivation.
Building on discussions with experts in the field and past reviews, we propose that the psychological acculturation experience can be understood in terms of behaviors, cognitions, emotions, and desires. Psychological acculturation in this framework might, for example, be understood or measured in terms of behavioral acculturation, such as language use, or voting; cognitive acculturation, such as ethnic identification, or cultural values endorsement; affective acculturation, such as feeling at home, or loneliness; motivational acculturation, such as the satisfaction of competence or independence needs; or as a complex combination of any or all of these aspects (also see Table \ref{tab:AspectExamples} for a range of example concepts). 

% Structure of Paper
In the following sections, we will develop the framework in more detail and will then apply it to a systematic review of past methodological, applied empirical, and theoretical literature on acculturation. This should also allow us to synthesise a status quo of the scattered literature and identify gaps within it.

\begin{table}%[hbt]
\caption{Examples of Coding Levels. }
\label{tab:AspectExamples} 
\begin{tabular}{>{\raggedright\arraybackslash}p{0.15\linewidth} 
>{\raggedright\arraybackslash}p{0.20\linewidth} 
>{\raggedright\arraybackslash}p{0.35\linewidth} 
>{\raggedright\arraybackslash}p{0.25\linewidth}}
\hline 
Aspect & Construct & Concept & Operationalization \\ 
\hline \\ [-0.5em]
Affect & 
Moods \linebreak Emotions \linebreak Feelings \linebreak & 
Loneliness \linebreak Feeling at home \linebreak Satisfaction with life \linebreak Pride \linebreak Comfortableness \linebreak Joy \linebreak Ease \linebreak Well-being \linebreak Worry \linebreak Trust \linebreak & 
"I feel ..." \linebreak "I enjoy ..." \linebreak "My mood ...." \linebreak\\

Behavior & 
Activities \linebreak Habits \linebreak Mannerisms \linebreak & 
Language use \linebreak Civic participation (voting, ...) \linebreak Performance (work, ...) \linebreak Media consumption \linebreak Education \linebreak Peer contacts \linebreak Food consumption \linebreak Cultural habits (holidays ...) \linebreak Delinquance \linebreak Marriage \linebreak & 
"I do ..." \linebreak "I speak ..." \linebreak "I meet ..." \linebreak \\ 

Cognition & 
Knowledge \linebreak  Memories \linebreak Evaluations \linebreak & 
Ethnic identification \linebreak Cultural values \linebreak Acculturation orientation \linebreak Preferences (food, friends, ...) \linebreak Knowledge \linebreak Importance ratings \linebreak Inner thought language \linebreak Perceived obligations \linebreak Beliefs \linebreak Stereotypes \linebreak & 
"I prefer ..." \linebreak "I think ..." \linebreak "I identify as ..." \linebreak \\ 

Desire & 
Needs \linebreak Goals \linebreak Wants \linebreak & 
Competence \linebreak Independence \linebreak Self-coherence \linebreak Belonging \linebreak Achievement \linebreak Justice \linebreak Growth \linebreak Respect \linebreak Acceptance \linebreak Identity continuity \linebreak & 
"I want ...",\linebreak "I would like to ..." \linebreak "I need ..." \linebreak \\ 

\hline
\end{tabular}
\end{table}


\section{Framework Development} 

This framework was developed based on collaborations with resettlement organizations and migrant citizens as well as based on past reviews and theoretical developments within the field. We will introduce each of these sources and the development in short before we discuss the structural elements in more detail and apply the framework to the systematic literature review.

One important note on our focus is the differentiation between dominant and non-dominant culture members. While psychological acculturation can apply to any inter-cultural contact \citep[including, for example, the contact between indigenous people and colonial settlers, e.g.,][]{Berry1974}, a majority of theories and studies have focused on cross-national immigrants. And while the framework we propose is applicable to psychological acculturation in its wider sense, its development and our focus in the following sections will lay mainly with the experience of non-dominant immigrants (as is common in the field).

\subsection{Focus Group Discussion}
In a first step we reached out to societal stakeholders of the acculturation process to gather information on key acculturation aspects in the lived realities out in the field. These experience reports formed the earliest foundation of our framework aspects. The focus group consisted of migrants, refugees, teachers, language- and integration coaches, volunteers and staff of a regional refugee resettlement agency as well as a representative of the local government. The 12 carefully selected and invited participants joined a 120 minutes round-table focus group discussion on the concept of acculturation. With informed consent by the participants, the discussion was audio recorded, transcribed, and later coded in three coding cycles of (1) initial In Vivo coding using the participants own words, (2) open coding inductively identifying common topics and elements, as well as (3) a content analysis using focused coding to summarize the overarching themes, which included the experience elements discussed here.

\subsubsection{Behavior}
As behaviors are often a particularly visible aspect of cultural interaction processes, the discussants elaborately discussed the social-behavioral acculturation aspects and expected behaviors in the new environment. The group particularly highlighted language learning and social contacts as examples of prominent behavioral acculturation aspects. Importantly, these behaviors were not only highlighted as a process towards acculturation but as an integral part of acculturation itself. For example, to have interactions outside the home was described as an important part of connecting to the culture and to being seen as an active member of society. Importantly, the group also discussed that many (interactive) behaviors are reciprocal and reactive in nature, depending on the collaboration with dominant cultural group members.
\begin{displayquote}
    Maria:\\
    {[...]} \textit{while, of course, you integrate best when you go to work.}
    
    Moderator:\\
    \textit{Why is that exactly?}
    
    Maria:\\
    {[...]} \textit{Because there you have daily contacts with locals.}
\end{displayquote}

\subsubsection{Cognition}
The round table discussion also highlighted the newcomer’s cognitive development as a key process of psychological acculturation. One aspect prominent during the discussions was that 'how we think about ourselves and the world' is often integral for navigating a new environment --- including, for example, language, social- and communication norms. A second major topic discussed was the role of identity development. The newcomers pointed to both a break in identity (the struggle of defining oneself in the new environment) as well as the struggle of dealing with a singular (migrant or refugee) identity label and the process of developing a more complex identity narrative towards others. Another key discussion point was that many behaviors are partly in service of cognitive acculturation. Participants, for example, highlighted that through interactions with dominant group members, migrants can learn about the social system, values, and social rules.
\begin{displayquote}
    Yahya:\\
    {[...]} \textit{Once I started my education, I felt part of society. That was actually a really big difference. That I have friends at my school. That is when ... I felt that I was a part of society. And that is a very big difference to before. That was not the case when I was learning Dutch at the university, at the language school, or at other places before. Only later when I was at school, ... I feel: Okay, now I feel I am really in the Netherlands.}
    
    Joop:\\
    \textit{Why did you not have that at the language school?}
    
    Yahya:\\
    \textit{Because I was only a refugee.}
\end{displayquote}

\subsubsection{Affect}
When the discussions turned to affective components of the acculturation process, the main and underlying sentiment was that the emotional aspects of acculturation foreground the importance of the subjective experience as acculturation. Instead of focusing on purely behavioral outcome conceptualizations (e.g., housing, job, education), the group highlighted the importance of considering the affective acculturation experience (e.g., feeling at home, feeling accepted). As an example, when asked why having a job is important to acculturation, newcomers and supporters pointed to a feeling of usefulness and being part of society. Similarly, the group pointed out that learning the local language was not only important to contact and communication but to feeling welcomed and judging the extend of acceptance. 
\begin{displayquote}
    Fariq:\\
    {[...]} \textit{But for me the language is very very difficult. And then you think people are not open. And you don't understand because your language isn't that good. And then you maybe don't feel welcomed when I have questions or want to approach them. But when you learn the language and then get into contact with people, ... then you know whether you are accepted or not.}
\end{displayquote}

\subsubsection{Desire}
The motivation aspect of acculturation was likely discussed in the most interesting way during the focus group discussion. The (lack of) motivation to interact with the new culture and its members was one key discussion point but motives for actions and psychological needs of the migrants were also discussed as more impalpable properties of other acculturation aspects (e.g., the need for acceptance during interactions). Yet, importantly the motivational aspect also highlighted the functional essence of individual acculturation. The needs for interactions, to be understood, for purpose, and for identity continuity were discussed as not only expected by the dominant cultural group but as intrinsic and fundamental to the health and functioning of the newcomers as part of the acculturation process.

\subsubsection{Complex Experience Process}
It remains important to point out that these themes emerged as part of the focus coding but were not discussed in isolation from one another. The group, for example, pointed out that most of the cognitive and affective acculturation aspects are often facilitated through (structured) social behavioral activities such as school, work, sports, music, and associations. Similarly, the feelings of becoming part of society (including descriptions of feeling useful, self-esteem, and purpose) were related to behavioral adaptation (contact, and work), and cognitive adaptation (developing a complex identity narrative beyond the refugee identity). Moreover, the group also highlighted during their discussions, that in practice acculturation was mostly a developmental or temporal process rather than a static end goal. While the group did discuss possible comparisons of how one person might be more acculturated than another, an overwhelming majority of the discussion centered around the acculturation process and aspects that are elementary to any migration experience. In sum, the voices of experts in the field and of those actively going through the migration process highlighted the importance of focusing on the common elements of individual experiences in order to comprehensively understand the elements of the concept. In our coding of the discussion we identified affect, behavior, cognition, and desire related conceptualizations of acculturation experience.

\subsection{Past Literature}
The focus group discussion, thus, inspired this undertaking and laid the groundwork for the development of an experience framework of psychological acculturation. Yet, at the same time, the framework also grew out of a strong theoretical tradition in the field and arguably brings together many of the past advances in capturing psychological acculturation.

% Experience Structure
While the focus group discussion offered a bottom up description of the individual acculturation experience aspects, there is also a growing theoretical consensus on the structuring elements of human experiences. A growing body of literature suggests that human experiences can fundamentally be understood as a set of needs, emotions, cognitions, and behaviors \citep[sometimes referred to as the ABCs or ABCDs of psychology: affect, behavior, cognition, desire; e.g.,][]{Cottam2010, Hogg2005, Jhangiani2014}.\footnote{It should also be noted that ABC(D) frameworks have been used effectively to structure theories and models across a wide variety of fields --- including research on attitudes \citep{Breckler1984} and ambivalence \citep{VanHarreveld2015}, self-regulation \citep{Ben-Eliyahu2015}, the big five personality traits \citep{Wilt2016}, suicidality \citep{Harris2015} and in clinical interventions \citep{Eifert1989}. Interestingly, the affect, behavior, and cognition structure has even found application in the development of human-like machines \citep{Guo2020}.} Following the premise that any human experience can be perceived within these four basic elements, we belief that the an ABCD framework of psychological acculturation could offer a comprehensive and theory-driven framework to structure and analyze the concept of psychological acculturation.

% ABC in Theoretical Traditions
Interestingly, this ABC structure is not entirely foreign to the the field of acculturation. Ward and colleagues \citep{Ward2001, Masgoret2006, Ward2019} have previously pointed out that theoretical perspectives on acculturation tend to focus on affect, behavior, or cognition (forming the ABCs of acculturation). Within the affective tradition Ward situates the stress and coping literature, behavioral traditions are the cultural learning theories, and social identification theories form the cognitive theories. \citet{Sam2006b} and others have even noted that such a perspective might be useful in structuring the core components of psychological acculturation. We agree and propose that, once we include desires (i.e., motivational literature), an expanded ABCD structure would offer a general framework of the concept because it comprehensively captures the fundamental aspects of the human experience.

In the next sections we will look at past literature on each of the aspects in more detail to discuss how the experience aspects relate to the concept of psychological acculturation. 

\subsubsection{Affect}
Affect entails the human capacity to feel \citep[including emotions and moods;][]{FeldmanBarrett2007}. These affective experiences profoundly differ between cultures \citep[e.g.,][]{Holodynski2012, Boiger2018} and inversely emotions are an integral part of cultures and their narratives \citep{Ahmed2014, Kitayama1994, Smith2016c, Sundararajan2015}. Their experience and expression is, thus, often a key aspect when two cultures come into contact \citep[e.g.,][]{Iyer2008, Stephan1992}. Moreover, in a wider sense, emotions have long been a gauge of adaptation to social change and major life events \citep[e.g.,][]{Smith1990, Pacella2017}. Phenomena such as culture shock \citep{Ward2001a} and homesickness \citep{VanTilburg1996} are common examples of how social change can have an emotional impact on well-being and might function as an indicator of adaptation. Previous literature has, thus, established a deep connection between affect and cultures, cultural contacts, as well as adaptation.

From these findings it is not surprising that affect and emotion have also been discussed within the psychological acculturation literature. \citet{Ward2001} in her review of the acculturation traditions, describes the stress and coping literature --- especially Berry's concept of acculturation stress \citep{Berry1997b} --- as the affect component of acculturation. In this tradition, the main constructs that constitute the affective dimension are the psychological and emotional well-being as part of the psychological adaptation process \citep[including, for example life satisfaction and depression][]{Ward2019}. However, beyond the theoretical stress literature tradition, there are also more immediate models and measurements of emotional acculturation. There is, for example, a relatively young tradition of 'emotional acculturation' as a distinct concept in which acculturation is understood as the similarity in emotional patterns \citep[see][for a review]{DeLeersnyder2017}. But also individual emotions, such as 'feeling accepted' \citep{Jasini2018}, or 'pride' \citep{Suinn1995} have received attentions as discrete aspects of acculturation. 

\subsubsection{Behavior}
Behaviors --- that is actions and mannerisms --- are often learned (i.e., extra-somatic or extra-genetic) and especially social behaviors are often culturally transmitted \citep{Legare2019, Whiting1980}. Cultural differences are, thus, often most visible in behavioral differences, including language use, dress- and food preferences, rituals and habits, as well as behavioral norms more generally \citep[e.g., social norms, as well as formal rules and laws; e.g.,][]{Hofstede2001}. And, inversely, behaviors are also an important aspect of what we consider culture \citep[e.g.,][]{Varnum2017}. One prominent recognition of this relationship between culture and behavior, has been the protection of indigenous cultural practices as a manifestation of their culture \citep[Art. 11]{UnitedNations2007}. Behaviors and their norms are, thus, a key component when cultures get into contact \citep[also see][]{Maxwell2017, Sam2010}. And also in a broader notion, behavioral changes have long been considered within the adaptation and well-being literature \citep[e.g.,][]{Luhmann2012}. In sum, behaviors lay at the very foundation of many cultural contacts and can be essential in forming an adaptive relationship with one's environment.

Given the overt nature of behaviors and their interconnectedness with culture, behaviors have also been a prominent aspect of the acculturation literature. Ward and colleagues \citeyear{Ward2019} in their review have identified cultural learning theories as one key literature tradition that has focused on behavioral aspects of acculturation. They relate these learning theories to the acquisition of effective skills and competences as the behavioral operationalizations \citep[including, verbal and non-verbal communication skills][]{Ward2001}. Other examples of behavioral conceptualizations of acculturation (not mentioned by Ward and colleagues), include civic participation \citep[e.g., voting;][]{Lessard-Phillips2020}, inter-ethnic marriage \citep[e.g.,][]{Song2009}, and media consumption \citep[e.g.,][]{Shoemaker1985}. 

\subsubsection{Cognition}
Cognitive aspects commonly entail the thinking processes of the human experience and thus include cultural knowledge, values, identities, beliefs, and attitudes, which are likely the most widely discussed aspects of non-material culture \citep[e.g.,][]{DiMaggio1997}. Yet at the same time cognitions are also substantially influenced by culture \citep[e.g.,][]{Gelfand2011, Nisbett2002}. As such, cognitions and their affordances often structure inter-cultural contacts \citep[e.g., values;][]{Ramstead2016} and contact perceptions \citep[e.g., out-group attitudes;][]{Stephan2000a}. Moreover, processes such as meaning making, self-image restoration, and dissonance reduction have established cognitive adaptation as a vital element in well-being and adaptation processes \citep[e.g.,][]{Czajkowska2017}. Cognitions are, thus, a prominent aspect of the human experience intimately intertwined with conceptions and influences of culture. Cognitions also structure and gauge cultural contacts and are a key process of human adaptation.

Given the pertinent connection between culture and cognitive processes, it might come as little surprise that cognitions have also played a major role in the acculturation literature. \citet{Ward2001} and colleagues have identified literature traditions on ethnic identity and group perceptions within the field. Within this cognitive tradition, Ward and colleagues particularly focus on Berry's \citeyear{Berry1997b} acculturation attitudes \citep{Ward2019}. Beyond the cultural attitudes tradition, the acculturation literature has recently also focused on several other cognitive conceptualizations of psychological acculturation, including cultural values \citep[e.g.,][]{Marin2003} and stereotypes \citep[e.g.,][]{Stanciu2018}. 

\subsubsection{Desire}
Desires here encompass the motivational forces of the human experience and there is substantial evidence that many of the psychological wants and needs are culture specific \citep[e.g.,][]{McInerney2016, Morling2017}. Yet, inversely the cultures themselves in part consist of motivational ideals \citep[or oughts; e.g., see][]{Markus1991}. Given this link to culture it is not surprising that psychological needs have recently also been highlighted as a key element in inter-group and inter-cultural relations \citep{Dovidio2017, Kitayama2007, Hassler2021, Shnabel2008a}. But also more broadly, needs have long been linked to different well-being regulations \citep[e.g.,][]{Steverink2006} and recently first evidence has emerged showing how motivations might fundamentally drive adaptation processes \citep{Dignath2020}. 
In sum, motivation has a deep connection to culture, cultural contacts, and psychological adaptation.

Yet, despite these connections, motivation is seldom discussed as a distinct aspect of the psychological acculturation concept. Few of the past reviews have examined its role within the literature or the concept \citep[including,][]{Ward2001a, Ward2019}. However, in recent years needs and wants have been discussed more frequently as a conceptual aspect of psychological acculturation --- with more researchers looking at migration driven by reason for migration \citep{Sandu2018}, as well as the motivations of acculturation orientations \citep{Recker2017a}, acculturation behavior \citep{Reece2000}, and psychological adaptation \citep{Safdar2003}. 

\subsubsection{Process}
%% Dynamic process rather than static end-product: 
% Experience can answer this call because it can only be understood based on past experiences
A final, fundamental factor we would like to address in the experience framework is the understanding of psychological acculturation as a dynamic process rather than a static end-product. That psychological acculturation is a developmental process, and that ``acculturation occurs when two independent cultural groups come into \textit{continuous first-hand contact over an extended period of time}'' \citep[][186]{Berry1989} seem to be a generally accepted assumption within the field. Yet, some reviews have pointed out that few applied studies have actually considered the theoretical implications of migration as a process and even fewer have methodologically followed the trajectories of migrants over time \citep[][]{Brown2011, Ward2019}. We believe that the experience framework of psychological acculturation, as it is presented here, is ideally suited to deal with this conceptualization as a developmental process. Philosophers of the phenomenological tradition have long highlighted that subjective experience can only be understood within the history of past experiences \citep[e.g.,][]{Heidegger1867}. 

\subsubsection{Experience elements and lived experience}
While we have introduced the four experience aspects as distinct elements it is important to note that both in theory and in practice affect, behavior, cognition, and desire are not experienced as distinct entities. This was already highlighted during the focus group discussion but is also reflected in theories on the aspects. As an example, most emotions have a cognitive component just as most cognitions have an emotional value. Similarly, motivation is commonly conceives as having both emotional (e.g., desire) and cognitive (e.g., goals) aspects, both of which are often directed towards behaviors (i.e., conation). Muddying the waters further is the difficulty that many operationalizations (and empirical measures) of psychological acculturation are complex concepts in themselves. Concepts such as satisfaction or distress, which are common measures of acculturation, famously include emotional and cognitive components. 

Yet, despite the interdependence of aspects and the complexity in the lived experience, the four elements can consistently be identified within experiences and concepts --- they remain qualitatively different aspects of the experience. And as such, they offer a pragmatic lens to structure the psychological acculturation concept \citep{Kuhn1962}. Differentiating the four (needing, feeling, thinking, doing) qualities of an experience in the what we consider psychological acculturation to be, allows us to structure our discussions of past, current, and future theories and measures of psychological acculturation.

% Applications
% relate back to problem: Theory and Application; Past and Future
Because the experience framework seeks to broadly define the aspects of the psychological acculturation \textit{concept} it arguably applies to several uses of the concept. Two key levels relevant to most researchers and practitioners are the abstract theory or model level and the more empirical measurement or intervention level. Questions about what we mean with psychological acculturation and how it relates to other concepts are likely different from questions of how we capture and assess the concept. Yet, the structuring properties of an ABCD experience framework apply to both levels and relate back to the current hurdles within the field.

Looking at the past literature, the ABCD structure allows us to asses and compare past theories, models, and measurements of the concept. Which aspects of the acculturation experience have been considered as part of the concept in theories, validated measures, empirical investigations, and interventions? And if multiple aspects have been considered, what are patterns of co-occurrences, how were the aspects related to one another, and how have they related to other concepts?

Looking at the status quo and future directions, the ABCD aspects allow us to make focus and selection decisions for theories, models, and measurements. Which aspect is relevant for a particular theory or measurement, and how do these elements relate to one another, as well as other concepts? And how do these elements develop over time in their relations with themselves and other aspects and concepts?

\begin{figure}[h]
\centering
\caption{Conceptual Model with Context.}
\includegraphics[width=\textwidth]{Figures/ConceptualFrameworkStatic.pdf}
\label{fig:ModelContext}
\end{figure}

\section{The Present Study}

%\begin{center}
%    \Warning\ \textsc{next sections not re-worked yet} \Warning
%\end{center}

The aim of our empirical efforts presented here is to put our proposed framework to the test. We have lamented that one of the challenges of a heterogeneous field is that it is difficult to assess and compare past literature. As a framework, we have suggested that the psychological elements of experiences could comprehensively structure our assessment of the literature. We will thus retrieve the past psychological literature that has proposed or used a measure of psychological acculturation as well as theories of psychological acculturation. For each, we will extract which experiential aspects were considered in the research. We expect that these efforts will provide insights into the perceived importance of desires, affects, cognitions, and behaviors for psychological acculturation. We also expect that this allows us to assess how many experience aspects are usually considered and which aspects are considered jointly. And finally, we aim to compare the understanding of psychological acculturation across different fields to assess the comparative utility. 

To apply the framework, we specifically target three bodies of literature that capture the concept of psychological acculturation. Firstly, we will assess methodological literature developing acculturation measures. As operationalizations of the concept within the empirical literature, validated scales usually focus on a concept in generalized manner, rather than focusing on aspects only relevant to a specific `applied' investigation. Coding psychological acculturation measures separately might also aid future considerations of measure selection because we effectively build a data base of scales that can be filtered by whether the scale includes measurements of affects, behaviors, cognitions, and desires. 
Secondly, once we have considered the validated scales in particular, we will more generally assess the empirical literature that measured psychological acculturation. Capturing operationalizations within more applied studies, allows us to investigate the focus within the empirical literature more broadly and allows us to compare differences between fields and research subjects.
Thirdly and finally, we will also probe the theoretical literature on psychological acculturation. The utility of a conceptual framework arguably lies in its applicability on all levels of the concept. Investigating the experience aspects considered across different theories of psychological acculturation, thus, not only offers insight into the focus and gaps within the literature but also grants an opportunity to apply the experience framework of psychological acculturation.

In the following section we will briefly discuss how we extract key information from the literature and will then sequentially analyze the role of experience elements in the methodological, applied empirical, and theoretical literature of psychological psychological acculturation\footnote{It should also be noted that we consciously chose not to conduct a meta-analysis. We conduct this review exactly because we are worried about comparability across studies, a key requirement of meta-analyses \citep{Pogue1998}. In our case we, arguably, do not have a clearly defined concept and exclusions to ensure a cohesive data set would be counter productive to our efforts. Moreover, a meta-analysis is commonly understood as an analysis of analyses \citep{Glass1976}. However, since we are interested in a conceptualization (rather than a relationship, a scale metric, or population parameter) a quantitative summary in form of a meta-analysis is not well-suited to answer our research question. Also a meta-analysis of our own extracted data seems profitless because it would likely mirror a sample size weighted average.}.

\section{Systematic Review}
% import methods and results from R Markdown (in file: "Methods-and-Results.tex")
\subsection{Methodological Literature}

Based on the systematic review and its coding, the first data set we
assess is a database of scale validations. We bring together the scales
suggested in previous reviews as well as validation studies we
identified in our own review. Throughout our literature review we found
five major works that reviewed the measurement of acculturation
\citep{Celenk2011, Maestas2000, Matsudaira2006, Wallace2010, Zane2004}.
After the removal of duplicate scales, we added any scale validation
that was present in our own systematic review but not included in the
previous reviews. For each measure we extracted the full item list as
well as the item scoring prior to coding. A comprehensive and
interactive database of the scales, with reference- and publication
information, as well as our experience elements and -context coding is
available in our online supplementary information as well as on our open
science repository (\hl{OSF and/or github citation here}).

\subsubsection{Methods}

Taken together these five reviews collected a total of 197 scales, of
which 75 were duplicates. From our own review we added 25 additional
validation studies. After removing duplicates this meant that we
considered a total of 122 unique scales for our coding. Of these scales
we ultimately had to exclude 41, because they were either not accessible
or did not fit the the topic of our review (see Table
\ref{tab:ScalesExclusion}). The scales had an average of \hl{X.XX} items
and \hl{X.XX} sub-scales. Most items were rated on a five-point
(\hl{XX.XX}\%) or four-point likert-type scale (\hl{XX.XX}\%), with only
\hl{X} scales including categorical ratings. About a fifth of scales
(20.4918\%) included majority group members in their validation studies.
The earliest included validation was from 1972 with a majority of scales
being validated around the turn of the 21\textsuperscript{st} century
and the latest included validation study in 2018.

\begin{table}

\caption{(\#tab:ScalesExclusion)Reasons for Exclusion}
\begin{tabular}[t]{lc}
\toprule
Exclusion Reason & Frequency\\
\midrule
items not accessible & 15\\
article not accessible & 4\\
chapter not accessible & 1\\
not acculturation & 1\\
not measured & 1\\
thesis not accessible & 1\\
\bottomrule
\end{tabular}
\end{table}


\subsubsection{Results}

For the literature on scale validations, we assessed both the role of
experience elements in the measures as well as contextual differences.

\paragraph{Experience}

With our main aim of examining the experience structure within the
scales, we examined whether scales included a specific experience
elements but also examined the used elements in their complex
combinations. In terms of general inclusion of elements, most studies
included a measure of cognition (89.66\%) and behavior (82.76\%),
whereas only roughly half the studies included a measure of affect
(55.17\%) and only a fourth of the scales included a measure of motives
(28.74\%). However, only a minority of scales included only a single
dimension. There were only 5 scales that exclusively relied on
cognitions (5.75\%) and 4 scales that measured only behaviors (4.6\%).
Yet, inversely, there were also only 13 scales that measured all four
dimensions (14.94\%). Most studies measured two (37.93\%) or three
(36.78\%) dimensions. A majority of scales either measured behavioral
and cognitive elements (26.44\%) or behavioral, cognitive, and affective
elements (26.44\%; also see Figure \ref{fig:ElementsScales} and Table
\ref{tab:ScaleElementCooccurrences}). Looking at the number of elements
measured together we also see substantial differences in what kind of
scales include a certain element. Scales that included cognitions
measured an average of 2.67 elements, scales measuring behavior, on
average, measured a 2.71, while scales that included affect measures had
a complexity average of 3.1 and scales measuring motivation even
measured an average of 3.4 scales. Thus, most scales measure multiple
dimensions, yet they focus on easily accessible dimensions (i.e.,
behavior and cognition), less of what is considered `less accessible' or
`subjective' (i.e., affect and desires). This is also visible in the
circumstance that there were no scales that exclusively measured
motivational or emotional adaptation (while this was the case for both
cognitions and behaviors). And if emotional or motivational aspects were
measured they were on average measured in scales that were already more
complex (i.e., included more experience elements).

\begin{figure}[h]
\centering
\caption{Scales: Bar graph of the experience element combinations.}
\includegraphics[width=\textwidth]{Figures/ABCDFreq-1}
\label{fig:ElementsScales}
\end{figure}

\begin{table}
\begin{minipage}[t][\textheight][t]{\textwidth}

\caption{\label{tab:ScaleElementCooccurrences}Scales: Element Co-occurrence Matrix}
\begin{tabular}[t]{lcccc}
\toprule
  & Affect & Behavior & Cognition & Desire\\
\midrule
Affect & 117 & 82 & 111 & 46\\
Behavior & 82 & 169 & 145 & 46\\
Cognition & 111 & 145 & 204 & 65\\
Desire & 46 & 46 & 65 & 68\\
\bottomrule
\end{tabular}
\end{minipage}
\end{table}


\paragraph{Context}

To gain a general understanding of contextual factors within the
validated studies, we also assessed cross-study patterns of cultural,
individual, situational, and process-related focus points.

\subparagraph{Country}

To assess the cultural contexts for which scales were validated we
assessed the migrants' countries of settlement as well as the countries
of origin. We found that most scales investigated a single host country
(\textit{N} = 81) and most investigated one country of origin
(\textit{N} = 68). There was 1 scale that was validated in two countries
and 1 that was validated in three countries. Additionally, there was a
single scale that was validated in a larger multi-national survey
context (i.e., with multiple host and origin countries). There was also
one study with two scales that were validated with the origin culture as
the starting point (i.e., single origin country, multiple host
countries). Looking at the country patterns, we found that an
overwhelming number of scales were validated within a U.S. American
settlement context (\textit{N} = 61). But also the remaining receiving
societies were mostly `western' countries (e.g., Canada, The
Netherlands, The United Kingdom, Israel, Australia) with only individual
scales for Taiwanese, Nepalese, or Russian settlement contexts. For the
migrant origin societies there was slightly more variation. There were a
few migrant groups that were investigated specifically (e.g., Mexico:
14, China:7, South Korea: 4), however most validation studies targeted
broader categories of migrants (any migrants: 11, Asian: 5, Hispanic: 9,
LatinX: 5). This also made it difficult to identify patterns of cultural
combinations investigated (apart from Mexican and LatinX migrants in the
United States).

\subparagraph{Sample}

To assess the role different groups of individuals targeted in the scale
validations, we coded the types of samples recruited for the validation
studies. A majority of studies sampled any consenting adult from the
migrant group of interest (\textit{N} = 50). As seems common in academic
research, a larger portion of the validated scales relied on young
migrants or students (\textit{N} = 29). Interestingly, only small
minority of validated scales targeted more vulnerable groups, such as
clinical samples (\textit{N} = 2) or refugees (\textit{N} = 2) --
despite a considerable focus on these groups within the broader
literature.

\subparagraph{Domains}

To assess the situational focus within the validated scales, we assessed
the number of domains within each scale as well as more common domains
across the scales. A relatively large number of scales asked about the
current state of the migrant in general manner without mentioning any
context or life domain (\textit{N} = 24; e.g., `'In general, in what
language do you read and speak?'`). The remaining scales referred to an
average of 3.49 dimensions (\textit{SD} = 3.42, range: 1 -- 21). A total
of 179 unique domains was measured across the 87 scales. The domains of
'language` (\hl{XX}\%), 'food' (\hl{XX}\%), `interactions' (\hl{XX}\%),
`family' (\hl{XX}\%) and `values' (\hl{XX}\%) were focused on most often
(see Online Supplementary Materials \hl{X}, Figure \hl{X}). Thus, while
there was large variation between the scales in the number, and
diversity of domains, the most frequently mentioned domains were in line
with the life domains proposed in the literature
\citep[e.g.,][]{Arends-Toth2007}.

\vspace{1em}
\todo[inline]{Should be re-coded to test our proposed domains. Also, re-check `general' code}

\subparagraph{Migration time}

All scales were validated using cross-sectional data after the migrant
arrived in the settlement society. This is in line with observations by
previous reviews of the field \citep[e.g.,][]{Brown2011}.

\subsection{Empirical Literature}

After analysis of the scales validations, we assessed the broader
empirical works we collected within the systematic review. We first
looked at all available empirical publications (incl.~books, chapters,
and dissertations). We later also assessed differences between fields
the work was published in. However, because we considered the fields on
an audience level, we used only empirical journal articles -- for which
journal-level audience data is available.

\subsubsection{Methods}

The search produced a total of \hl{XXX} results to which we added
\hl{XX} articles through contacts with experts in the field. We
subsequently screened out results that did not fit into our review.
After duplicate removal (\(N_{excluded}\) = \hl{XX}, \(N_{screened}\) =
484), we excluded 92 results in the title screening as well as an
additional 126 results during the abstract screening. Of the remaining
266 results, 259 papers presented empirical work on acculturation and
were coded. The 7 non-empirical results were reviews, which were not
coded because they did not fit into our coding schema. During the full
text coding we excluded an additional 26 results because they were
either not relevant or were not accessible (for exclusion reasons see
Table \ref{tab:EmpiricalExclusion} and for our PRISMA diagram see Figure
\ref{fig:PRISMA}).

\begin{figure}[h]
\centering
\caption{PRISMA diagram. Position still undecided. Currently generated in R based on n(row) maybe make prettier. \Warning\ Re-check numbers before duplicates removed and number of papers added from other sources.}
\includegraphics[width=\textwidth]{Figures/PRISMA}
\label{fig:PRISMA}
\end{figure}

\begin{table}
\begin{minipage}[t][\textheight][t]{\textwidth}

\caption{\label{tab:EmpiricalExclusion}Exclusion Reasons Empirical Literature}
\begin{tabular}[t]{lccc}
\toprule
\multicolumn{1}{c}{ } & \multicolumn{3}{c}{Screening} \\
\cmidrule(l{3pt}r{3pt}){2-4}
Exclusion Reason & Title & Abstract & Full Text\\
\midrule
not acculturation & 225 & 116 & 11\\
not migrant & 65 & 41 & 6\\
not migration & 62 & 42 & 7\\
not ABCD & 29 & 42 & 5\\
not English & 1 &  & \\
not measured &  & 32 & 33\\
items not accessible &  &  & 36\\
thesis not accessible &  &  & 32\\
article not accessible &  &  & 4\\
book not accessible &  &  & 4\\
review &  &  & 3\\
chapter not accessible &  &  & 2\\
should still be coded &  &  & 1\\
\bottomrule
\end{tabular}
\end{minipage}
\end{table}


Of the final works we coded, 192 were journal articles, 37 theses, and 4
book chapters. Most studies presented quantitative data (\textit{N} =
205), mixed methods (\textit{N} = 14), or qualitative data (\textit{N} =
11), while the remaining 3 manuscripts were reviews of empirical data. A
vast majority of the authors used the term `acculturation' (or
derivative versions, such as `acculturation attitudes' or `acculturation
orientation'; \textit{total N} = 178), or `integration' (\textit{N} = 7)
to refer to cultural adaptation. Notably, a majority of the empirical
investigations did not share common measures of cultural adaptation --
186 studies used measures that were reported a maximum of five times,
with a considerable majority of papers using new or ad-hoc measures of
cultural adaptation. Only about every tenth study included the local
majority in the study (\textit{N} = 25, 10.8225\%). Cultural adaptation
most frequently was a predictor variable (\textit{N} = 99, 42.8571\%), a
dependent variable (\textit{N} = 72, 31.1688\%), or a correlation
variable (\textit{N} = 27, 11.6883\%) in the empirical works. This
pattern was mirrored when looking at the focus of the papers, where a
majority of the papers had acculturation as their main focus (\textit{N}
= 83, 36.7257\%), with other bodies of work focusing on health outcomes
(\textit{N} = 23, 10.177\%), or inter-group relations (\textit{N} = 12,
\texttt{empFocusRelationPerc}\%) as their main outcomes. The earliest
included study was published in 1970, with a continuous increase of
publications after the year 2000, with considerable publication peaks in
2011 and 2019. We provide full descriptions of descriptive data
extractions and additional information about the data description in
Online Supplementary Information X.

\paragraph{Field of Publication}

For the broader empirical literature, we also collected additional data
on the field the studies were published in. To assess the differences
between fields we merged the `Scimago Journal Ranking Database'
\citep{SCImago2020} with our empirical review. For all available journal
articles we added information on key journal metrics (incl.~H index,
impact factor, and data on the field and audiences). This also meant
that dissertations, book chapters, and books were excluded from this
analysis because data on their publishers is not readily available or
unreliable. Additionally, 8 journals were not included in the Scimago
database (likely because they do not have an ISSN identifier or were
discontinued before 1996, see Online Appendix \hl{X}, Table \hl{X} for
the missing journals). We ultimately had journal metrics for 183
empirical articles. The Scimago database classifies each journal
according to the field(s) that the journal aims to address. Importantly,
(1) each journal can be be classified to address multiple fields and (2)
the field include codes of fields (e.g., `Social Sciences') as well as
sub-fields (e.g., `Social Psychology'). This leads to the case that
there can be substantial overlap between fields, and journals cannot
easily or readily be assessed in mutually exclusive subgroups.

To summarize the articles further we then classified the field
combinations into super-ordinate discipline codes. These discipline
codes are based in part on U.S. Department of Education's subject
classifications \citep[i.e., CIP;][]{InstituteofEducationSciences2020},
the U.K. academic coding system
\citep[JACS 3.0;][]{HigherEducationStatisticsAgency2013}, the Australian
and New Zealand Standard Research Classification
\citep[ANZSRC 2020;][]{AustralianBureauofStatistics2020}, as well as the
Fields of Knowledge project \citep{ThingsmadeThinkable2014}. We
ultimately classified each journal into one of four mutually exclusive
disciplines (psychology: \textit{N} = 61, multidisciplinary: \textit{N}
= 57, Medicine, Nursing, and Health: \textit{N} = 51, and Social
Sciences (miscellaneous): \textit{N} = 14. For a full discussion of the
classifications see Online Supplementary Materials \hl{X}).

\subsubsection{Results}

To test the utility of our framework, we again assessed the role of
experience elements in the measurement as well as contextual
differences.

\paragraph{Experience}

In terms of the overall frequencies of experience elements, the broader
empirical data mirrored that of the validation studies. Most studies
included a measure of cognition (83.26\%) and behavior (80.69\%),
whereas only about half of all studies included a measure of affect
(51.93\%) and only a fifth of the studies included a measure of motives
(18.45\%). Yet, only 42 studies focused on a single element
(\(N_{behavior\ only}\) = 20, \(N_{cognition\ only}\) = 18,
\(N_{emotion\ only}\) = 4). Similarly, only 17 papers included measures
of all four experience elements (7.3\%). Most studies measured three
(37.77\%) or two dimensions (36.91\%). Different from the scale
validations, within the broader empirical works most works included
measures of emotions, behaviors, and cognitions (\textit{N} = 69,
29.61\%), with a further substantial number of articles measuring
behaviors and cognitions (\textit{N} = 53, 22.75\%. Also see Figure
\ref{fig:EmpPlotFreq-1} and Table
\ref{tab:EmpiricalElementCooccurrences}). Looking at the number of
elements measured together we again see substantial differences in what
kind of scales include the individual elements. Scales that included
cognitions measured an average of 2.53 elements, scales measuring
behavior, on average, measured a 2.52, while scales that included affect
measures had a complexity average of 2.86 and scales measuring
motivation even measured an average of 3.23 scales. Thus, interestingly,
not a single study measured only motivation, and measures of motives
remained mostly limited to scales that were already multidimensional.
The results exacerbate the pattern found in the scale validations,
complex measures and conceptions of acculturation are seen infrequently
and readily accessible (i.e., less subjective) dimensions of cognition
and behavior remain the focus of most studies.

\begin{figure}[h]
\centering
\caption{Empirical Literature: Bar graph of the experience element combinations.}
\includegraphics[width=\textwidth]{Figures/EmpPlotFreq}
\label{fig:EmpPlotFreq-1}
\end{figure}

\begin{table}
\begin{minipage}[t][\textheight][t]{\textwidth}

\caption{\label{tab:EmpiricalElementCooccurrences}Empirical Literature: Element Co-occurrence Matrix}
\begin{tabular}[t]{lcccc}
\toprule
  & Affect & Behavior & Cognition & Desire\\
\midrule
Affect & 258 & 208 & 236 & 55\\
Behavior & 208 & 433 & 339 & 74\\
Cognition & 236 & 339 & 422 & 76\\
Desire & 55 & 74 & 76 & 90\\
\bottomrule
\end{tabular}
\end{minipage}
\end{table}


\paragraph{Context}

To gain a general understanding of contextual factors within the broader
empirical studies, we again assessed cross-study patterns of cultural,
individual, situational, and process-related focus points.

\subparagraph{Country}

To assess the cultural contexts on which the authors focused, we again
assessed the migrants' countries of settlement as well as the countries
of origin. Similar to the validations, an overwhelming number of scales
were validated within a North American settlement context (United
States: \textit{N} = 109, Canada: \textit{N} = 26). But also the
remaining receiving societies were mostly `western' -- Western Europe
(e.g., The Netherlands, United Kingdom, Germany, Italy, Spain),
Australasia (Australia, New Zealand), Russia, and Israel. And only 10
studies focused on data from multiple receiving societies.

When it came to the migrants' country of origin, a majority of studies
were indifferent to migrants background and simply recruited any
consenting migrant (\textit{N} = 37), or recruited a category of
migrants (e.g., LatinX or Hispanic: \textit{N} = 21, African: \textit{N}
= 10). Among the individual countries target there was a particular
focus on the east and south-east Asian region (e.g., China: \textit{N} =
21, South Korea: \textit{N} = 19, Vietnam: \textit{N} = 11). Yet,
different from the scale validations, there was a large variety of
different origin countries that were included in less than five studies
(\textit{N} = 101 regions were targeted less than five times). Thus, the
receiving countries mainly mirrored those for which scales were
validated, yet there was an extensive number origin countries which were
investigated individually or migrants were considered irrespective of
their cultural origin. Moreover, it was again not possible to identify
distinct cultural adaptation clusters within the literature (that would
be large enough to compare).

\subparagraph{Sample}

To assess the role different groups of individuals targeted in the
empirical work, we again coded the types of samples recruited for the
studies. A majority of studies sampled any consenting adult from the
migrant group of interest (\textit{N} = 145, 62.23\%). Of the targeted
sampling strategies, most recruited refugees (\textit{N} = 22, 9.44\%),
young migrants (\textit{N} = 20, 9.01\%), or elderly people (\textit{N}
= 14, 6.01\%). The remaining fifth of the studies recruited other more
specific samples (e.g., nurses, athletes, Muslims). Interestingly,
despite the circumstance that a large portion of papers focused on
mental health outcomes, only 7 studies (3\%) recruited clinical migrant
samples. These results speak to the case that relatively few empirical
studies actually take individual differences into account in their
sample selection. Studies may still address individual differences
within the data description and within the modeling approaches (e.g.,
controlling for gender), yet it seems that inter-sectional
idiosyncrasies did not seem to play a major role in the targeting of
samples.

\subparagraph{Domains}

To capture the situational focus of the authors, we coded which life
domains the utilized measures referred to. A relatively large number of
studies assess cultural adaption in general manner without mentioning
any context or life domain (\textit{N} = 116). The remaining studies
referred to an average of 2.16 dimensions (\textit{SD} = 2.69, range: 1
-- 21)). We found a total of 183 unique domains across the 233 studies.
The dimensions of `language` (\hl{XX}\%), 'food' (\hl{XX}\%), `social
interactions,' and `friends' (\hl{XX}\%) were included most frequently.
So, while larger proportion of studies ask about cultural adaption in
general (outside of a specific domain), the number of domains included
and addressed is extensive and diverse. The mentioned domains at times
go beyond the life fields mentioned in previous work (also see Online
Supplementary Materials \hl{X}).

\vspace{1em}
\todo[inline]{Should be re-coded to test our proposed domains. Also, re-check `general' code}

END SECTION


\section{Discussion}

\subsection{Summary [not an actual heading; for structure only]}

% Literature Levels
Compare Theory - Scale - Application levels --- see Figures \ref{fig:CombinedAspects}, \ref{fig:CombinedComplexity}, \ref{fig:CombinedAspectComplexity}.


% Recap: Problem, Aim, Proposal
% Problem:
An enormous variety of aspects of our lives are affected by cultures, the psychological changes we experience when we get into continuous first-hand contact with another culture (i.e., psychological acculturation) are consequently equally plentiful and diverse.
% Aim:
In order to make sense of past theories and measures of psychological acculturation and to develop new theories and measures, it is thus necessary to build a conceptual framework that allows us to analyze, compare, and understand the individual aspects of psychological acculturation.
% Proposal:
In this paper, we have proposed that taking the fundamental aspects of the human experience (affect, behavior, cognition, and desire) would offer a comprehensive and theory-based structure to the psychological acculturation concept (in both theory and application).

% Focus Group Discussion:
Our investigation has utilized a variety of empirical sources and applications that offer support for the applicability of an experience framework in the acculturation field. Firstly, the ABCD experience framework is, in part, based on a qualitative investigation, where key societal stakeholders have clearly distinguished behavior, cognition, affect, and desire as important aspect of acculturation. The discussants highlighted that behavioral and cognitive aspects have more expectations from host population and are more visible, but affect and desire are equally important and often form powerful motors of change. The discussion of psychological acculturation in the field has thus offered a strong bottom-up source for the validity of the framework and the importance of all four aspects.

% Literature Review:
% Scales, Empirical, Theories
And secondly, we also applied the experience-based framework in a systematic review of past methodological, applied empirical, and theoretical literature on psychological acculturation. In the screening exclusion reasons for all three bodies of literature, we find that virtually all investigations fit within the ABCD experience framework. The investigations of acculturation that did not fit within the ABCD framework assessed acculturation by exclusively collecting information that did not need the agency of the migrant, such as length of residence, or migration status. The circumstance that we were able to identify different experience aspects in such an overwhelming majority of psychological acculturation literature might offer a first indicator for the applicability of the framework.

Looking at the methodological literature that did assess the acculturation experience, we find a wide variety of complexities between validated scales. Almost all combinations of affect, behavior, cognition, and desire had corresponding scales and most scales included two or three experience aspects. Among the different experience elements, behavioral and cognitive aspects of acculturation were measured most often and were also assessed most often together. The more internal aspects of affective and motivational acculturation were assessed less commonly and if they were assessed they were often found in scales that also included cognitions or behaviors. This was in stark contrast to behaviors and cognitions, which often would form scales without assessing any other aspect of the experience.

The broader empirical literature that had assessed psychological acculturation within their research, mostly mirrored the patterns of the validated scales in an exacerbated form. Proportionally, even fewer studies assessed desires and few studies assessed all four aspects of the experience. Additionally, the validated scales were not readily used but authors often relied on shortened, modified or ad hoc measures. Looking at differences between fields we also find quantitative and qualitative differences. We find that psychological journals tended to publish papers that included significantly more experience aspects than did other fields. And we also find that there are substantial differences in the relative importance of individual aspects. While affective and motivational acculturation are consistently assessed least frequently, the relative importance of cognitions and behaviors were able to differentiate the fields from one another.

Finally, looking at the theoretical literature, we find that ...

\subsubsection{Features [not an actual heading; for structure only]}
In sum, we thus find that the experience framework showed a number of useful features, which address past conceptual issues. 

Firstly, the experience approach is based on basic human faculties, that is generally speaking every healthy person has the capacity for emotions, thoughts, desires, and behaviors. Focusing on fundamental faculties relevant to any conscious person makes the framework widely applicable across cultural contexts. ABCD frameworks have, for example, been found across cultures \citep[e.g.,][]{Bhawuk2011} and likely structure human fore brain functioning \citep{Swanson2020}\footnote{It is important to note that while anyone will have motives, emotions, thoughts, and behaviors, what one needs (e.g., belongingness or independence), feels (e.g., sadness or happiness), thinks (e.g., identification or disinterest), or does (e.g., studying or working) is highly ideographic. It is this ideographic content that makes the framework relevant to such a broad range of migration contexts. Yet it is the content-free structure --- the presence or absence of the basic elements in conceptualizations of acculturation --- that is transferable across contexts and studies, enabling comparisons and broader conceptual discussion. It should also be noted that in our opinion such a framework does not stand in conflict with cultural or indigenous psychological concerns of an absolutist, or deterministic psychology \citep[e.g.,][]{Kim2006a}. In fact, cultural psychologists, together with many decolonial researchers, have long argued that the individual embedded and lived experience should gain a more central role in our theoretical developments \citep[e.g., ontological turn;][]{Pedersen2020}}. It is thus not surprising that we were able to identify ABCD aspects in acculturation research from more than 40 cultural contexts.

Secondly, because affect, behavior, cognition, and desire broadly capture the human experience, the experience framework comprehensively captures the psychological element of acculturation. The framework, thus, captures a broad and complex phenomenon while still offering a clear and theory-driven structure of the concept and its applications. This meant in our application that very few studies did not capture any experience aspect and we were arguably able to make meaningful comparison across a wide variety of contexts and even fields.

Thirdly, experiences are also relevant across time-scales. In most cases experiences rely on past experience and collectively generate the present experience \citep[also see][]{Husserl1959, Heidegger1867}. As such, experiences are scale-able evaluations --- of a single situation, a recent period, or a life-long journey \citep[e.g.,][]{Clewett2019}. There is usually a clear 'temporality' to human experiences and experiences can focus on the past, present, and even the future. We see this in the focus group discussion as well as in the systematic review, where the framework was able to capture cross-sectional as well as longitudinal studies and was relevant with samples prior and post the migration event (including, prospective migrants). 

Finally, the experience conceptualization of psychological acculturation is inherently a bottom-up approach to the topic. Taking migration experiences as the starting-point highlights the considerations for the lived realities of the researched individuals and communities. Scholars in the traditions of critical research methods have long highlighted the importance of including the participants in the research conceptualization process \citep[e.g.,][]{Kovach2009}. If one uses the experiences of the researched individuals to guide the study questions and design, one inevitably emphasizes the agency and needs of the community -- lending relevance and ownership of knowledge to the community \citep[e.g., ][]{Schmidt2021}. This was also one of the main focus points of the focus group discussion and was visible in the theoretical literature.

\subsection{Limitations}
Yet, the framework and this study are not without limitations. Notably, the framework exclusively focuses on the psychological acculturation process. This has been the explicit focus of our efforts but this also means that non-psychological aspects such as biological, cultural, or societal changes are not captured directly but only to the extent to which they impact the experiences of the involved people. So while sociologists or political scientists might aim to capture the group-level changes more specifically, or medical professionals might want to assess physiological changes, this is not the focus of psychological acculturation as it is structured here. 

Another point that we have thus far mostly ignored is the role of the migration context. While we would argue that the framework structure (i.e., the four experience aspects) is relevant across any context, the perceptions of an experience are often fundamentally influenced by the context and environment of the experience. We elaborate on the contextual factors in more detail within Online Supplementary Material A, but we would like to briefly discuss the contextual embeddedness of the acculturation experience. 

Which perceptions a given context permits (i.e., contextual affordances) has long been a topic of discussion in the field of ecological psychology \citep[e.g., see][]{Cantor1994}. Recent efforts to formalize the structure of contexts have, for example, highlighted the interactive process of group-environments and individuals \citep[e.g.,][]{Young2002}.
Adjacent to these discussions in ecological psychology, we would like to address three contextual factors in particular: (1) Cultures, (2) individuals, and (3) the situations.

\paragraph{Culture}
% Culture as social facts and Cultural adaptation as tension between different social facts.
The most prominent contextual factor of psychological acculturation is probably the concept of culture. Functional structuralists, who tried to structure the concept of culture, have defined culture as external social expectations focused on our "manners of acting, thinking and feeling" \citep[][p. 52; on social facts]{Durkheim1982}. In this understanding, cultures are the external counter-part to experiences and define expected patterns of behavior (e.g., dress or communication styles), cognition (e.g., sense of race-, class-, gender-, and sexual identities), emotions (e.g., expressions of emotions), and motivations (e.g., virtues and duties). In the case of psychological acculturation, an individual needs to deal with (at least) two sets of cultural expectations --- their heritage culture and the culture of the new environment \citep[e.g., see models of][]{Berry1997b, Berry2006a}. The individual will thus have to negotiate their individual response to the cultural expectations by their heritage and host culture.

Once we consider culture as external influences on emotions, behaviors, cognitions, and desires and psychological acculturation as the individual adjustments to the presence of multiple cultural influences, it becomes apparent that the experience framework may be ideally suited to discuss the concept of culture. The psychological acculturation experience (i.e., the individual experience of ABCD) allows one to capture the adjustment to tension as a result of differing external expectations of affect, behavior, cognition, and desire. Studying psychological acculturation in the experience framework then also allows us to reflect on which cultures' social expectations are considered in the conceptualization of acculturation \citep{Bhatia2001}. 

\paragraph{Individual}
%% Individual based on inter-group contact and Berry: 
% Individual differences in general (e.g., age, gender) but also migration related differences (e.g., reason for migration, language proficiency)
Another contextual factor to consider during the psychological acculturation process are the interacting individuals themselves. There has been a rising focus on the idea that acculturation centers around the daily interpersonal interactions a person has with people of the other group \citep{Maxwell2017, Sam2010}. And although it can, at times, be difficult to disentangle cultural from individual influences, there is a range of personal features that likely influence the psychological acculturation process. These personal differences might relate to relatively stable individual differences, such as gender or personality, but also migration related differences, such as the reason for migration (e.g., voluntary vs. forced migration), cultural distance, or migration status. Within the migration related factor, we would also include aspects that might change over the course of the adaptation process but give migrants different starting positions, such as language skills and education level. Similar to culture, individual differences likely play a role for multiple aspects of the psychological acculturation process (also see Figure \ref{fig:ModelContext}).

\paragraph{Situation}
%% Situations as domains of psycho-social functioning: 
% Many theories have come up with life domains that form different cultural interaction situations.
Beyond the cultural group and the individuals, the interactions of psychological acculturation are further dependent on the situational context. One way of structuring this situational context is the idea that the social experience will take place within different domains in life. A wide variety of theories have proposed different situational domains that are relevant to social and cultural interactions. What structurally unites these conceptualizations of situational domains is the dimension of closeness to the individual. That is, most areas of life found in the literature can be arranged from the most immediate (i.e., micro or private, such as family) to the broadest levels (i.e., macro or public, including government or media, also see Figure \ref{fig:ModelContext}). 

\subsection{Implications}
The experience framework of psychological acculturation as it is presented here, then has a few broader implications for practitioners and researchers. 

%% Use of the framework:
\paragraph{Practice} The framework might be of interest for practitioners and policy-makers because it is not only theory-based and brings together a wide range of past literature. Rather, the framework grew out of a grass-roots initiative and the personal experience puts the individual at the center of consideration. The bottom-up approach might thus be useful in making clear and informed decisions while still considering the concept in its personal complexity. As such the four experience aspects might offer a useful starting point for building interventions as well as structuring monitoring and evaluations efforts. And while oversimplified in this manuscript, the consideration of the individual experience might also offer a useful starting point to the contextual factors one might want to consider in policies and other interventions.

\paragraph{Research} For advancing future research projects on psychological acculturation the systematic review hints at the possibility that the framework might offer a tool to describe and compare measures, empirical work, and theories. Looking backwards, the framework could allow researchers to systematically compare conceptual differences for a specific topic. This would, for example, allow meta-analyses that assess a relationship of psychological acculturation with other concepts to ensure a clearer comparability between studies\footnote{Note that this was not an option in our case because we assessed acculturation across a variety of different fields that did not share other concepts or analyses.}. Looking forwards, the framework allows researchers to make informed decisions on which aspects are relevant for a novel study. In the application of theories (e.g., acculturation orientation) researchers would have a tool to decide which aspects of the experience are relevant to assess (across or within specific contexts).

% Novel predictions: Thinking about ABCD aspects allows for new questions about focus, connections, and differences. Why is A, B, C, or D more important? What is the relationship between A, B, C, and D? Are internal aspects (D) more important than external aspects (B)?
Moreover, the structural framework also allows for novel questions and predictions. The structure, for example, allows for considerations of the relative importance, the interconnections, or the accessibility of these elements. There might, for example, be theoretical reasons why a certain aspect is more important to a specific issue (e.g., the role of emotions and cognitions to mental health; e.g., \citealp{Crocker2013}). One could also offer process considerations, making theoretical arguments why a certain aspect might precede another or how one aspect might feed back into another (e.g., motivations guiding cognition and affect, which in turn drive behavior. E.g., theory of reasoned goal pursuit; \citealp{Ajzen2019}). Another example, for a novel prediction could be a consideration of the accessibility of the individual aspects, where more public aspects, such as behaviors, might be seen as more important to the dominant cultural group, whereas more internal aspects such as desires and goals might be more important to the newcomer (or their group).

However, there might also be structural patterns one could assess and novel questions one can ask. One might, for example, assess which emotions were most frequently included in measures and definitions of acculturation \citep[e.g., specific emotions such as anger or pride, but also types of emotions, such as positive or negative, or about yourself or others;][]{DeLeersnyder2017}. One might also be interested in co-occurences of contents (e.g., within aspect: identifications and attitudes, but also between aspects: the need for belongingness and emotion of loneliness). It would also allow for conscious considerations of relationships within or between experience aspects. And these types of questions can be asked about past definitions and measurements (i.e., are there patterns in the literature), for a specific field (e.g., which emotions are relevant in the clinical context) or a specific cultural context (e.g., are there specific needs of Syrian refugees in Europe). 

\subsection{Conclusion [not an actual heading; for structure only]}
A framework for psychological acculturation based on core aspects of experiences might thus offer a useful tool for both researchers and practitioners. The framework offers a theory-based way of comprehensively assessing, comparing, and understanding acculturation. We bring together insights from experts in the field as well as a wide array of past advances in the literature.

\newpage
\section{Parking Spot}
Don't know where this goes:

%% Host-Migrant Interconnectedness:
% Focus on migrant perspective because often disadvantaged position but any interdependent aspect should show up in ABCD + framework can also be used for majority perspective (but not focus of this paper)
Before we move on to the systematic review and our application of the conceptual framework, we would like to remark on our focus on the migrant's rather than the dominant cultural group member's experiences. 
While this framework explicitly focuses on the migrant's experience as the structuring experience, this is not meant to put the sole responsibility of ``migration success'' on the newcomer or ignore the interconnectedness of intercultural exchange. We focus on the migrant experience as the structuring element because migrants usually find themselves in disadvantaged positions. 

We would also like to address the issue of interactive acculturation. \citet{Bourhis1997a} and others, have often highlighted that cultural adaptation is essentially interactive and one should not consider the acculturation process of one individual or group without the consideration of the other group and the interaction partners. We agree, and while we address some of these issues explicitly in the framework, we would also like to highlight that within the experience framework aspects of cultural adaptation that are interconnected, co-dependent, or depend on the dominant group members likely influence the individual's migration experience. As an example, if a migrant is frequently confronted with exclusion and hostility, this will likely affect their motivations, emotions, thoughts, and behaviors. 
We would also like to highlight that the framework is equally applicable to the cultural adaptation experience of dominant cultural group members in their interactions with or perceptions of migrants. Although not the focus of this paper, the same assessments of measures, definitions, and understandings should be able to be conducted with the experience of the dominant group.
Additionally, this framework is not meant as an exhaustive or exclusive list of all adaptation aspects but rather offers a structural framework to analyze and understand the broad experience of cultural adaptation.
%Migration is a complex trans-formative experience within different areas of personal- and social life. 

\section{Notes to self}
\begin{itemize}
  \item assess domains based on framework we propose
  \item life domains (in part) based on Humanitas
\end{itemize}

\printbibliography

\appendix

\section{Search Strategy}
\label{app:AppendixSearchStrategy}

We performed our main literature searchers for the empirical literature on March 4\textsuperscript{th}, 2020 and February 14\textsuperscript{th}, 2021. The second literature search included alternate terms used less frequently to describe what we mean with psychological acculturation, including "transculturation" and "cultural transition". Additionally, the second search removed limiter terms that could have exclude interdisciplinary investigations and focused on human participants.
In designing our search strategy we used an adapted version of the `SPIDER' research tool \citep[e.g.,][]{Cooke2012}. We utilized the \textit{Evaluation} element mainly to exclude articles that were not relevant to the search. The exact search terms used are listed in Table \ref{tab:SearchStrategiesTab} below.

\begin{table}%[hbt!]
\caption{Search Strategy Cultural Adaptation Review}
\label{tab:SearchStrategiesTab} 
\begin{tabular}{ll}
\hline
Element & Search Terms \\ 
\hline \\ [-0.5em]

% Sample
Sample & 
    (Immigration OR migration OR migrant OR immigration OR refugee) \\ 
    \\ [-0.25em]
    
% Phenomenon of Interest
\begin{tabular}[t]{@{}l@{}}Phenomenon \\of Interest\end{tabular} & 
    \begin{tabular}[t]{@{}l@{}}(acculturation OR enculturation OR transculturation OR \\
    assimilation OR "social integration" OR "cultural adaptation" OR \\
    "cultural adjustment" OR "cultural transition")\end{tabular}  \\ 
    \\ [-0.25em]
    
% Design
Design & 
    \begin{tabular}[t]{@{}l@{}}("measurement tool" OR scale OR instrument OR \\
    questionnaire OR survey OR definition OR inventory)\end{tabular}  \\ 
    \\ [-0.25em]
    
% Evaluation
Evaluation & 
    \begin{tabular}[t]{@{}l@{}}NOT (parent* OR college OR resilience OR treatment OR \\
    intervention OR therapy)\end{tabular}  \\ 
    \\ [-0.25em]
    
% Research Type
Research type * & 
    (quantitative OR qualitative OR "mixed method") \\ [0.75em] 
    \hline

% Note
\multicolumn{2}{l}{* This element was ultimately dropped because it was too sensitive in PsycInfo.}
\end{tabular}
\end{table}


\end{document}

%% 
%% Copyright (C) 2019 by Daniel A. Weiss <daniel.weiss.led at gmail.com>
%% 
%% This work may be distributed and/or modified under the
%% conditions of the LaTeX Project Public License (LPPL), either
%% version 1.3c of this license or (at your option) any later
%% version.  The latest version of this license is in the file:
%% 
%% http://www.latex-project.org/lppl.txt
%% 
%% Users may freely modify these files without permission, as long as the
%% copyright line and this statement are maintained intact.
%% 
%% This work is not endorsed by, affiliated with, or probably even known
%% by, the American Psychological Association.
%% 
%% This work is "maintained" (as per LPPL maintenance status) by
%% Daniel A. Weiss.
%% 
%% This work consists of the file  apa7.dtx
%% and the derived files           apa7.ins,
%%                                 apa7.cls,
%%                                 apa7.pdf,
%%                                 README,
%%                                 APA7american.txt,
%%                                 APA7british.txt,
%%                                 APA7dutch.txt,
%%                                 APA7english.txt,
%%                                 APA7german.txt,
%%                                 APA7ngerman.txt,
%%                                 APA7greek.txt,
%%                                 APA7czech.txt,
%%                                 APA7turkish.txt,
%%                                 APA7endfloat.cfg,
%%                                 Figure1.pdf,
%%                                 shortsample.tex,
%%                                 longsample.tex, and
%%                                 bibliography.bib.
%% 
%%
%% End of file `./samples/longsample.tex'.
