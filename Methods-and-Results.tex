\subsection{Methodological Literature}

Based on the systematic review and its coding, the first data set we
assess is a database of scale validations. We bring together the scales
suggested in previous reviews as well as validation studies we
identified in our own review. Throughout our literature review we found
five major works that reviewed the measurement of acculturation
\citep{Celenk2011, Maestas2000, Matsudaira2006, Wallace2010, Zane2004}.
After the removal of duplicate scales, we added any scale validation
that was present in our own systematic review but not included in the
previous reviews. For each measure we extracted the full item list as
well as the item scoring prior to coding. A comprehensive and
interactive database of the scales, with reference- and publication
information, as well as our experience elements and -context coding is
available in our online supplementary information as well as on our open
science repository (OSF and/or github citation here).

\subsubsection{Methods}

Taken together these five reviews collected a total of 197 scales, of
which 75 were duplicates. From our own review we added 25 additional
validation studies. After removing duplicates this meant that we
considered a total of 122 unique scales for our coding. Of these scales
we ultimately had to exclude 41, because they were either not accessible
or did not fit the the topic of our review (see Table
\ref{tab:ScalesExclusion}). The scales had an average of \hl{X.XX} items
and X.XX sub-scales. Most items were rated on a five-point
(\hl{XX.XX}\%) or four-point likert-type scale (\hl{XX.XX}\%), with only
\hl{X} scales including categorical ratings. About a fifth of scales
(20.4918\%) included majority group members in their validation studies.
The earliest included validation was from 1972 with a majority of scales
being validated around the turn of the 21\textsuperscript{st} century
and the latest included validation study in 2018.

\begin{table}

\caption{(\#tab:ScalesExclusion)Reasons for Exclusion}
\begin{tabular}[t]{lc}
\toprule
Exclusion Reason & Frequency\\
\midrule
not migration & 14\\
items not included & 8\\
search pending & 5\\
not accessible & 4\\
not found & 3\\
not acculturation & 2\\
majority focussed & 1\\
not found probably the same as Tsai et al. 2000 & 1\\
only language (no scale) & 1\\
same as S-029 & 1\\
uses other scale & 1\\
\bottomrule
\end{tabular}
\end{table}


\subsubsection{Results}

For the literature on scale validations, we assessed both the role of
experience elements in the measures as well as contextual differences.

\paragraph{Experience}

With our main aim of examining the experience structure within the
scales, we examined whether scales included a specific experience
elements but also examined the used elements in their complex
combinations. In terms of general inclusion of elements, most studies
included a measure of cognition (XX\%) and behavior (XX\%), whereas only
roughly half the studies included a measure of affect (XX\%) and only a
fourth of the scales included a measure of motives (XX\%). However, only
a minority of scales included only a single dimension. There were only 5
scales that exclusively relied on cognitions (XX\%) and 4 scales that
measured only behaviors (XX\%). Yet, inversely, there were also only 13
scales that measured all four dimensions (XX\%). Most studies measured
two (XX\%) or three (XX\%) dimensions. A majority of scales either
measured behavioral and cognitive elements (XX\%) or behavioral,
cognitive, and affective elements (XX\%). Thus, most scales measure
multiple dimensions, yet they focus on easily accessible dimensions
(i.e., behavior and cognition), less of what is considered `less
accessible' or `subjective' (i.e., affect and desires). This is also
visible in the circumstance that there were no scales that exclusively
measured motivational or emotional adaptation (whereas that was the case
for both cognitions and behaviors).

END SECTION
