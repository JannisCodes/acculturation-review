\newcommand{\nTheo}{92}
\newcommand{\nMeth}{233}
\newcommand{\nEmp}{526}

To assess the past empirical and theoretical literature on psychological
acculturation, we performed a systematic review. We first read seminal
and review works within the field
\citep[including,][]{Ward2019, Berry1997b, Berry2003, Szapocznik1978, Sam2006a, Rudmin2003a}.
Based on our reading of the literature, we designed a comprehensive
literature search strategy in an iterative fashion. For the empirical
work on acculturation we performed a literature search on March
4\textsuperscript{th}, 2020 and February 14\textsuperscript{th}, 2021,
within the ``APA PsycINFO'' bibliographic databases using the
EBSCO\textit{host} provider. The databases also included the
PsycARTICLES, PsycBOOKS, and PsycCRITIQUES databases as well ProQuest
Dissertations with psychological relevance (for the full information on
the search strategy see Appendix \ref{app:AppendixSearchStrategy}).

Together with past reviews, we used this literature search to identify
validated scales as well as empirical works more generally. For the
theoretical literature we collected the theories used in the empirical
works and performed an additional, more specific, search of the same
databases as well as the Web of Science Core Collection using the
Clarivate Analytics provider on March 3\textsuperscript{rd}, 2021 (for
full information see Appendix \ref{app:AppendixSearchStrategy}).

From the literature searches we created three separate databases of
theoretical, methodological, and applied empirical works on
psychological acculturation. For each literature search, we downloaded
all references and abstracts, which two independent coders screened for
relevance after duplicate removal --- first based on the titles and then
based on the abstracts. We downloaded all relevant and available works
for full-text coding. For all three types of works we extracted a range
of variables to apply our framework. The full coding process and data
extraction are described in the online Supplemental Material A and B
\citep[also see our repositories;][]{Kreienkamp2021d, Kreienkamp2021e}.

\subsection{Theoretical Literature}

The most abstract level of our review was concerned with how researchers
conceptualized psychological acculturation in their theoretical work.
Our theory-specific literature search produced a total of 477 results
from which we identified 73 theories. From our review of the empirical
literature, we added an additional 19 theories (total N = 92, for
exclusion reasons, see Table \ref{tab:ExclusionsCombined} and for our
PRISMA diagram see Figure \ref{fig:PrismaCombined} A. A full table of
all theories, with references, and final coding is available in our
Supplemental Materials as well as on our open science repository
\citep[see][]{Kreienkamp2021d, Kreienkamp2021e}.

\begin{figure}[h]
\centering
\caption{PRISMA Diagrams for the Theoretical, Methodological, and Empirical Literature.}
\makebox[\textwidth]{\includegraphics[width=\paperwidth, trim={0 0 0 1.5cm}]{Figures/PrismaCombined}}
\label{fig:PrismaCombined}
\end{figure}

\begin{table}
\begin{minipage}[t][\textheight][t]{\textwidth}

\caption{\label{tab:ExclusionsCombined}Exclusion Reasons for all Literature Levels}
\begin{tabular}[t]{lccccccc}
\toprule
\multicolumn{1}{c}{ } & \multicolumn{3}{c}{Theoretical} & \multicolumn{1}{c}{Methodological} & \multicolumn{3}{c}{Empirical} \\
\cmidrule(l{3pt}r{3pt}){2-4} \cmidrule(l{3pt}r{3pt}){5-5} \cmidrule(l{3pt}r{3pt}){6-8}
Reason & Title & Abstract & Full Text & Full Text & Title & Abstract & Full Text\\
\midrule
not English & 5 & 1 & 1 & 1 & 1 &  & \\
not migration & 45 & 3 & 1 &  & 62 & 42 & 7\\
not migrant & 24 & 11 & 4 & 1 & 65 & 41 & 6\\
not acculturation & 49 & 17 & 16 & 1 & 225 & 116 & 11\\
not ABCD & 7 & 1 &  &  & 29 & 42 & 5\\
not theory & 20 & 71 & 25 &  &  &  & \\
not measured &  &  &  & 1 &  & 32 & 35\\
items not accessible &  &  &  & 16 &  &  & 36\\
thesis not accessible &  & 1 &  & 1 &  &  & 33\\
article not accessible &  &  &  & 1 &  &  & 4\\
book not accessible &  &  &  &  &  &  & 4\\
chapter not accessible &  &  &  & 1 &  &  & 2\\
poster not accessible &  & 1 &  &  &  &  & \\
should still be coded &  &  &  &  &  &  & 1\\
\bottomrule
\end{tabular}
\end{minipage}
\end{table}


\subsubsection{Methods}
\paragraph{Dataset}

The authors of the 92 included theoretical works self-categorized their
contributions as a theoretical conceptualization (\textit{N} = 9),
theoretical framework (\textit{N} = 26), theory (\textit{N} = 36), or
theoretical model (\textit{N} = 21). And while 29 authors explicitly
targeted a specific part of acculturation (e.g., 7 identity
acculturation theories and 4 labor market acculturation theories), a
majority of theoretical works offered commentary on the overall
construct of acculturation (N = 63). Looking at the types of theory
building, a majority of proposals were purely theoretical (N = 75) with
the remaining theoretical works growing out of qualitative
investigations (such as grounded theory approaches; N = 17).

\paragraph{Experience Aspects}

To assess the experience aspects that were considered as part of the
theoretical works, two independent coders coded the authors' axioms,
theorems, and model elements for self-identified inclusions of affects,
behaviors, cognitions, and desires (all inter-rater agreements were
96.74\% or above and all Cohen's \(\varkappa\)s were above 0.82,
\(\varkappa_{pooled}\) = 0.94; for full inter-rater reliability see
Supplemental Material B). We only coded explicit mentions by the authors
and we did so on three different levels. An example of these three
levels for affect would be phrases of ``mood'' or ``emotions''
(construct level), ``anxiety'' or ``pride'' (concept level), and ``the
migrant feels \ldots{}'' (operationalization level). A list with further
examples can be found in Table \ref{tab:AspectExamples} and a full
description of the coding is available through our coding protocol (see
Supplemental Material A).

\paragraph{Process}

To assess the focus on psychological acculturation as a process or an
outcome, we coded whether authors self-identified the theory as a
process (e.g., `process', `development', `longitudinal', `temporal',
`dynamic') or an outcome (e.g., `static', `outcome', `markers',
`consequence').

\subsubsection{Results}

Our main goal was to assess the use of the four affect, behavior,
cognition, and desire elements within the theoretical conceptualizations
of psychological acculturation. Looking at the overall usage of the
experience aspects we find that virtually all theoretical works included
behavioral aspects (94.57\%; e.g., cultural practices, media
consumption) and a vast majority considered cognitive aspects (90.22\%;
e.g., navigation knowledge, ethnic identification). We found
considerably less mentions of affective (46.74\%; e.g., anxiety, pride)
and motivational aspects (41.3\%; e.g., independence goals, need to
belong). But the generally high usage of the aspects, also meant that
only about a tenth of the theories focused on a single aspect (6.52\%).
Interestingly, all theories that considered only one aspect were
exclusively focusing on behaviors (N = 5) or cognitions (N = 1). Of the
remaining theories, 21 (i.e., 22.83\%) considered all four aspects,
leaving a majority of theoretical works to considered two aspects
(36.96\%) or three aspects (33.7\%). Among these, the most common
combinations of experience aspects were behavioral and cognitive
acculturation (28.26\%) or behavioral, cognitive, and motivational
aspects combined (17.39\%; also see Figure \ref{fig:ElementsTheories}
and Table \ref{tab:CombinedCooccurrences}).

Looking at the number of aspects considered together we also see
substantial differences in what kind of theories include a certain
aspect. Theories that included behaviors considered an average of 1.78
other aspects (\textit{SD} = 0.78), and theories considering cognitions,
on average, also included 1.87 other aspects (\textit{SD} = 0.65).
Theories that included the more internal aspects of affect or desire
showed a considerably higher number of additional aspects considered
(affect: \textit{M} = 3.35, \textit{SD} = 0.52; desire: \textit{M} =
3.50, \textit{SD} = 0.36). Thus, most scales measure multiple dimensions
of acculturation (\textit{M} = 2.73, \textit{SD} = 0.79; also see Figure
\ref{fig:LiteratureComparison}). Yet they tend to focus on more external
aspects of behavioral and cognitive acculturation, and less on internal
aspects of affects and desires. This is also visible in the observation
that there were no theories that exclusively focused on emotional or
motivational acculturation while this was the case for both cognitions
and behaviors. And if emotional or desire aspects were considered they
were found in theories that tended to already include a higher number of
other experience aspects.

\color{blue}

To assess the process focus of the theoretical works, we assessed
whether authors self-identified their works as a process or outcome
focused. We found that 49 of the 92 coded theoretical works proposed
dynamic conceptualizations of psychological acculturation (53.26\%).
This slight majority is a notably high percentage, considering that past
reviews of the acculturation literature have pointed to a small number
of studies actually offering dynamic tests of theories
\citep[e.g.,][]{Brown2011, Ward2019}.

\begin{center}\rule{0.5\linewidth}{0.5pt}\end{center}

While it is beyond the scope of this paper to comprehensively summarize
and integrate the over 90 theoretical works on psychological
acculturation, we will briefly discuss how the different types of
theoretical works fit within a broader ABCD process. To this aim, we
break up the acculturation process into three functional steps and
highlight some works in their use of the affect, behavior, cognition,
desire components.

A broader pattern that emerged during our reading and coding of the
literature, is that theories focused on different phases of the
acculturation process. We believe that these phases can be meaningfully
separated into (1) acculturation conditions, (2) acculturation
responses, and (3) acculturation outcomes. This approach explicitly
builds on previous step-wise conceptualizations
\citep[e.g., process vs. outcome distinction][]{Sam2006b} and step-wise
models of acculturation
\citep[e.g.,][]{Arends-Toth2006a, TeLindert2008a}.\footnote{Alternative labels of the three stages would be "Immersion, Processing, Change" or "Evaluation, Negotiation, Navigation" to reflect some of the older theories that have proposed step-wise approaches \citep[e.g.,][]{Cross1991, Atkinson1993, Gordon1964a}.}
However, within our proposed ABCD framework, these three steps describe
the affects, behaviors, cognitions, and desires that are (1) normatively
accepted by the relevant cultural groups prior to the contact, (2)
actually experienced during the inter-cultural contact, either in person
or through media, institutions, or cultural products, and (3)
experienced after the contact (also see Figure \ref{fig:ModelContext}).

This is to say that theoretical works have, generally speaking, focused
on the socio-structural and personal expectations of the acculturation
experience
\citep[i.e., conditions; e.g.,][]{Kim1988, Rogler1994, Navas2005, Giles1977, Tartakovsky2012, Robinson2019, Ward2016, Serdarevic2005},
on the contact experience itself
\citep[i.e., response; e.g.,][]{Berry1992, Berry2005, Sam2003, Riedel2011, Ward2016}
or on the affect, behavior, cognition, and desire consequences of the
contact
\citep[i.e., outcome; e.g.,][]{Baird2015, Berry1998, Berry1992, Berry2005, Riedel2011, Rogler1994, Luedicke2011}.
\footnote{It is important to mention that several theoretical works have focused on process models that span two or more of the three steps \citep[e.g.,][]{Berry1992, Ward2016, Arends-Toth2006a, Rogler1994}.}

An additional functional benefit of the condition-response-outcome
separation has been the observation that several theoretical works sit
at the intersection between two of the steps. Several works were
specifically focused on the conditions under which acculturation
conditions lead to tension and changes in the acculturation responses
\citep[i.e., conditions of change; e.g.,][]{Masgoret2006, Alitolppa-Niitamo2004, Grove1985, Wood2014},
or under which conditions acculturation responses lead to stressful or
adaptive acculturation outcomes
\citep[i.e., conditions of stress; e.g.,][; also see Figure \ref{fig:ModelContext} and Figure S1]{Ryan2008, Berry1992, Benet-Martinez2005, Salo2015, Wood2014, Hajro2019}.

\begin{center}\rule{0.5\linewidth}{0.5pt}\end{center}

When looking at the conceptualizations more deeply we find that within
most theoretical conceptualizations of psychological acculturation
authors tended to discuss their focus on affect, behavior, cognition, or
desire aspects explicitly.

For example, \citet[][]{Bhawuk2006} calls for ``affective, behavioral
and behavioral training approaches'' to intercultural trainings
\citep[][Table 30.3]{Bhawuk2006} and \citet[][]{Robinson2019} focus on
``the values, beliefs, and behaviours (VBBs)'' (p.~133) in their
Cultural Congruence Process theory.

In a more recent work from the expatriate literature,
\citet[][]{Maertz2016} discusses affect, behavior, cognition, and desire
throughout their paper but one prominent illustration can be found when
they define their `process foundations': ``(a) automatic and controlled
cognitive processing are each important; (b) specific motives or goals
are important for processing social-emotional information; (c) Person /
Situation interactions are vital for how people think, feel, and respond
in the moment; and (d) emotion has a close interactive influence with
cognitive processing.'' (in-text citations removed for brevity).

We find similarly explicit discussion in the classic work of
\citet[][]{Berry1992}, where he divided psychological acculturation into
behavioral shifts and acculturation stress. Berry then goes on to write
that behavioral shifts are ``changes in behaviour {[}\ldots{]} and
include values, attitudes, abilities and motives'' (p.~70), and
acculturation stress is particularly manifested ``as lowered mental
health status (particularly anxiety, depression), feelings of
marginality and alienation, and heightened psychosomatic and
psychological symptom level''. In Berry's (1992) theorizing behavioral
shifts and acculturation stress jointly form `psychological
acculturation' and are proceeded by a process of adaptation, which
includes the famous acculturation strategies.

One exemplary theory-building effort is arguably \citet[]['s]{Kim1988}
work on the
\textit{Integrative Theory of Communication and Cross-Cultural Adaptation}.
Not only does Kim an excellent job a discussing boundary conditions, and
defining terms in the onset of her work. But her theory is stringently
built by explicitly formulating assumptions, axioms, theorems; as well
as embedding the theory within broader meta-theoretical works (e.g.,
General Systems Theory). And while she discusses the role of affects,
behaviors, cognitions, and desires throughout her theory-building, her
introduction of the communication competence concept offers an exemplary
illustration of her explicitness, when she writes that ``a number of
cognitive, affective and behavioral elements of host communication
competence are presented for their direct relevance to the present
theoretical concern, cross-cultural adaptation.''

\begin{center}\rule{0.5\linewidth}{0.5pt}\end{center}

Kim2019: Importance of goals and motivations prior and during migration
for down-stream adaptation processes Mchitarjan2013: one key determinant
of ``success or failure'' are ``motivational factors, i.e.~the motives,
desires, or goals of the minority and majority'' Madison2006: role of
motives

\begin{center}\rule{0.5\linewidth}{0.5pt}\end{center}

\begin{itemize}
\tightlist
\item
  one example of single aspect focused theorizing
  \citep[e.g.][]{Weinreich2009}
\end{itemize}

\color{black}

\begin{figure}[h]
\centering
\caption{Theoretical Literature: Bar graph of the experience element combinations.}
\includegraphics[width=\textwidth]{Figures/TheoriesFreq-1}
\label{fig:ElementsTheories}
\end{figure}

\begin{table}
\begin{minipage}[t][\textheight][t]{\textwidth}

\caption{\label{tab:CombinedCooccurrences}Aspect Co-occurrences for all Literature Levels}
\begin{tabular}[t]{lcccc}
\toprule
\multicolumn{1}{c}{Aspect} & Affect & Behavior & Cognition & Desire\\
\midrule
\addlinespace[0.3em]
\multicolumn{5}{l}{\textbf{Theoretical}}\\
\hspace{1em}Affect & N = 38 & -0.01 & 0.11 & 0.23*\\
\hspace{1em}Behavior & 35 & N = 85 & -0.11 & 0.15\\
\hspace{1em}Cognition & 35 & 74 & N = 81 & 0.23*\\
\hspace{1em}Desire & 20 & 35 & 35 & N = 36\\
\addlinespace[0.3em]
\multicolumn{5}{l}{\textbf{Methodological}}\\
\hspace{1em}Affect & N = 116 & -0.12 & 0.18** & 0.24***\\
\hspace{1em}Behavior & 82 & N = 167 & -0.10 & -0.03\\
\hspace{1em}Cognition & 107 & 141 & N = 191 & 0.13\\
\hspace{1em}Desire & 39 & 39 & 50 & N = 53\\
\addlinespace[0.3em]
\multicolumn{5}{l}{\textbf{Empirical}}\\
\hspace{1em}Affect & N = 258 & -0.04 & 0.28*** & 0.11*\\
\hspace{1em}Behavior & 208 & N = 433 & -0.10* & 0.00\\
\hspace{1em}Cognition & 236 & 339 & N = 422 & 0.05\\
\hspace{1em}Desire & 55 & 74 & 76 & N = 90\\
\bottomrule
\end{tabular}
\end{minipage}
\end{table}


\subsection{Methodological Literature}

Based on the systematic review and its coding, the first empirical
dataset we assess is a database of scale validations. We bring together
the scales suggested in previous reviews as well as validation studies
we identified in our own review. Throughout our literature review, we
found five major works that reviewed the measurement of acculturation
\citep{Celenk2011, Maestas2000, Matsudaira2006, Wallace2010, Zane2004}.
After removal of duplicate scales, we added any scale validation that
was present in our own systematic review but not included in the
previous reviews. For each measure, we extracted the full item list as
well as the item scoring prior to coding. A comprehensive and
interactive database of the scales, with all available items, reference-
and publication information, as well as our experience elements and
-context coding are available in Supplemental Material D as well as on
our open science repository
\citep[see][]{Kreienkamp2021d, Kreienkamp2021e}.

\subsubsection{Methods}  
\paragraph{Dataset}

After duplicate removal, these five reviews collected a total of 97
scales. From our own review, we added 159 additional validation studies
(total of 256 unique scales). Of these scales, we ultimately had to
exclude 23, because they were either not accessible or did not fit the
topic of our review (see Table \ref{tab:ExclusionsCombined}). About a
quarter of scales (24.22\%) included majority group members in their
validation studies. The earliest included validation was from 1948 with
a majority of scales being validated around the turn of the
21\textsuperscript{st} century and the most recent included validation
study was published in 2020.

\paragraph{Experience Aspects}

We extracted data on the experience aspects by primarily focusing on the
measured concepts and their operationalizations (also see Table
\ref{tab:AspectExamples}). For each article, we retrieved the items used
and coded whether the measure included references to affects, behaviors,
cognitions, and desires. Because this concerned the most central aspect
of our framework, each manuscript was double-coded and inconsistent
codes were resolved after discussion (all inter-rater agreements were
97.85\% or above and all Cohen's \(\varkappa\)s were above 0.95,
\(\varkappa_{pooled}\) = 0.96; for full inter-rater reliability see
Supplemental Material B).

At this stage, we also noted if items measured concepts that relate to
multiple experience aspects. As an example, a single item asking about
`satisfaction with the new life' might include emotional and cognitive
elements. In this case, we code the manuscript as measuring both
emotions and cognitions, but note that these elements are not measured
independently. We also noted if the measures do not consider an
individual's experiences, such as reporting migration status or length
of residency.

\paragraph{Process}

To extract an indicator of whether the scales were aimed at
psychological acculturation as a process or an outcome, we collected
information on assessed migration times (e.g., pre-migration,
post-migration) and the the validation type (e.g., cross-sectional,
longitudinal).

\paragraph{Context}

We additionally coded a range of contextual variables, including the
type of sample collected (e.g., student, clinical), the life domain
targeted (e.g., school, family), the cultures considered (i.e., host and
migrant groups), the type of analysis conducted (e.g., correlation,
regression), the measurement type (e.g., continuous scale, categorical
classification), as well as the variable type of acculturation in the
analysis (e.g., predictor, dependent). Further information as well as
analyses of the contextual variables go beyond the scope of the
immediate framework validation and are thus only presented in
Supplemental Materials B and C.

\subsubsection{Results}

With our main aim of examining the experience structure within the
scales, we examined whether scales included a specific experience
element but also examined the used elements in their complex
combinations. In terms of general inclusion of elements, most studies
included a measure of cognition (87.55\%) and behavior (72.53\%),
whereas only roughly half the studies included a measure of affect
(50.21\%) and only a fourth of the scales included a measure of desires
(29.18\%). However, only a minority of scales included only a single
aspect. There were only 18 scales that exclusively relied on cognitions
(7.73\%) and 21 scales that measured only behaviors (9.01\%). Yet,
inversely, there were also only 35 scales that measured all four aspects
(15.02\%). Most studies measured two (38.63\%) or three (27.9\%)
aspects. A majority of scales either measured behavioral and cognitive
aspects (23.61\%) or behavioral, cognitive, and affective elements
(19.31\%; also see Figure \ref{fig:ElementsScales} and Table
\ref{tab:CombinedCooccurrences}).

Looking at the number of aspects measured together we also see
substantial differences in what kind of scales include a certain aspect.
Scales that included cognitions also measured an average of 1.57 others
aspects (\textit{SD} = 0.77), scales measuring behavior, on average,
also included 1.62 other aspects (\textit{SD} = 0.77). Scales measuring
affect or desire measures included substantially more other aspects.
Scales that included affect measures also included 2.04 other aspects
(\textit{SD} = 0.61) and scales measuring desires even measured an
average of 2.31 other aspects per scale (\textit{SD} = 0.66; also see
Figure \ref{fig:LiteratureComparison}). Thus, most scales measure
multiple dimensions (\textit{M} = 2.39, \textit{SD} = 0.91), yet they
focus on external accessible aspects of psychological acculturation
(i.e., behavior and cognition), less of what is considered `internal' or
`subjective' (i.e., affect and desires). And if affect or desire
elements are considered, they often only occur in scales that already
include a higher number of other aspects. This is further underscored by
the observation that there were only 3 scales that exclusively measured
emotional acculturation and not a single scale that exclusively focused
on motivational acculturation (while this was the case for both
cognitions and behaviors).

\begin{figure}[h]
\centering
\caption{Methodological Literature: Bar graph of the experience element combinations.}
\includegraphics[width=\textwidth]{Figures/ABCDFreq-1}
\label{fig:ElementsScales}
\end{figure}

To assess the process focus of the scales we also assessed the migration
time the scale validators considered. Except for a single scale that was
validated for potential migrants, all scales were validated using
cross-sectional data after the migrant arrived in the settlement
society. This is in line with observations by previous reviews of the
field \citep[e.g.,][]{Brown2011}.

\subsection{Empirical Literature}

At the most applied level, we assessed the broader empirical studies.
This final database included the largest number of manuscripts and is in
theory the application of the theoretical and methodological literature.
The search produced a total of 1629 results to which we added 133
articles through contacts with experts in the field and from referenced
works within the review. After duplicate removal, title--, abstract--,
and full-text screening we coded a total of 526 empirical works (for
exclusion reasons see Table \ref{tab:ExclusionsCombined} and for our
PRISMA diagram see Figure \ref{fig:PrismaCombined} C).

\subsubsection{Methods}

\paragraph{Dataset}

Of the final works we coded, 452 were journal articles, 68 theses, and 6
book chapters. Most studies presented quantitative data (\textit{N} =
464), mixed methods (\textit{N} = 39), or qualitative data (\textit{N} =
20), while the remaining 3 manuscripts were reviews of empirical data.
Notably, a majority of the empirical investigations did not share common
measures of acculturation --- 391 studies used measures that were
reported a maximum of five times. A considerable majority of papers with
uncommon measures used new or ad-hoc measures of acculturation. Less
than a fifth of studies included local majority group members in the
study (\textit{N} = 77, 14.69\%). Acculturation most frequently was a
predictor variable (\textit{N} = 285, 54.39\%), a dependent variable
(\textit{N} = 148, 28.24\%), or a correlation variable (\textit{N} = 37,
7.06\%) in the empirical works. This pattern was mirrored when looking
at the focus of the papers, where a majority of the papers had
acculturation as their main focus (\textit{N} = 153, 29.48\%), with
other bodies of work focusing on health outcomes (\textit{N} = 163,
31.41\%), or inter-group relations (\textit{N} = 18, 3.47\%) as their
main outcomes. The earliest included study was published in 1948, with a
strong increase of publications after the year 2000, and a peak of
publications in 2012. We provide full descriptions of data extractions
and additional information about the data description in Supplemental
Material B.

\paragraph{Experience Aspects}

Extraction of the used experience aspects mirrored the methodological
literature assessment and we primarily focused on the measured concepts
and their operationalizations (also see Table \ref{tab:AspectExamples}).
The only exception were qualitative studies, which we coded following
the same codebook of the theoretical literature. All aspects were coded
by two independent coders (all inter-rater agreements were 97.91\% or
above and all Cohen's \(\varkappa\)s were above 0.93,
\(\varkappa_{pooled}\) = 0.97; for full inter-rater reliability see
Supplemental Material B) and inconsistencies were resolved after
discussion.

\paragraph{Process}

To assess the static or dynamic conceptualization of the empirical
studies, we again collected information on assessed migration times
(e.g., pre-migration, post-migration) and additionally coded the type of
data collected and analyzed (e.g., cross-sectional, longitudinal data
and data analysis).

\paragraph{Context}

Contextual variables on sample, life domain, cultural groups,
measurement type, and variable type are again beyond the scope of the
framework validation but are presented in Supplemental Material C.

\paragraph{Field of Publication}

For the broader empirical literature, we also collected additional data
on the field the studies were published in. To that end, we merged the
`Scimago Journal Ranking Database' \citep{SCImago2020} with our
database. For all available journal articles, we added information on
key journal metrics (incl.~H index, impact factor, and data on the field
and audiences). This also meant that dissertations, book chapters, and
books were excluded from this analysis because data on their publishers
is not readily available or unreliable. Additionally, 19 journals were
not included in the Scimago database (because they do not have an ISSN
identifier or were discontinued before 1996, see Online Appendix B for
the missing journals). We ultimately had journal metrics for 425
empirical articles.

To summarize the journal data we then classified the journal fields into
super-ordinate discipline codes. These discipline codes are based in
part on the U.S. Department of Education's subject classifications
\citep[i.e., CIP;][]{InstituteofEducationSciences2020}, the U.K.
academic coding system
\citep[JACS 3.0;][]{HigherEducationStatisticsAgency2013}, the Australian
and New Zealand Standard Research Classification
\citep[ANZSRC 2020;][]{AustralianBureauofStatistics2020}, as well as the
Fields of Knowledge project \citep{ThingsmadeThinkable2014}. We
ultimately classified each journal into one of four mutually exclusive
disciplines (psychology: \textit{N} = 122, multidisciplinary: \textit{N}
= 102, Medicine, Nursing, and Health: \textit{N} = 144, and Social
Sciences (miscellaneous): \textit{N} = 45. For a full discussion of the
classifications see Online Supplemental Material B).

\subsubsection{Results}

We assessed the role of experience aspects in the measurement and then
compared differences between fields.

In terms of the overall frequencies of experience elements, the broader
empirical data mirrored that of the methodological literature. Most
studies included a measure of cognition (81.75\%) and behavior
(80.23\%), whereas only about half of all studies included a measure of
affect (49.05\%) and less than a fifth of the studies included a measure
of desires (18.63\%). Yet, only 126 studies focused on a single
experience aspect (\(N_{behavior\ only}\) = 73, \(N_{cognition\ only}\)
= 47, \(N_{emotion\ only}\) = 6). Similarly, only 46 papers included
measures of all four experience aspects (8.75\%). Most studies measured
three (36.12\%) or two aspects (31.18\%; \textit{M} = 2.30, \textit{SD}
= 0.86). Different from the scale validations, within the broader
empirical works, most works included measures of emotions, behaviors,
and cognitions (\textit{N} = 158, 30.04\%), with a further substantial
number of articles measuring behaviors and cognitions (\textit{N} = 107,
20.34\%. Also see Figure \ref{fig:EmpPlotFreq-1} and Table
\ref{tab:CombinedCooccurrences}). Looking at the number of aspects
measured together we again see substantial differences in what kind of
scales include the individual aspects. Scales that included cognitions
measured an average of 1.54 other aspects (\textit{SD} = 0.68), scales
measuring behavior, on average, measured 1.48 other aspects (\textit{SD}
= 0.82), while scales that included affect measured an average of 1.97
other experience aspects (\textit{SD} = 0.43) and scales measuring
desires even measured an average of 2.27 other experience aspects
(\textit{SD} = 0.61; also see Figure \ref{fig:LiteratureComparison}).
Thus, not a single study measured only motivational acculturation (i.e.,
desires), and measures of desires remained mostly limited to scales that
were already measuring many of the other experience aspects. The results
exacerbate the pattern found in the scale validations, complex measures
and conceptions of acculturation are seen infrequently and external
aspects of cognition and behavior remain the focus of most studies.

\begin{figure}[h]
\centering
\caption{Empirical Literature: Bar graph of the experience element combinations.}
\includegraphics[width=\textwidth]{Figures/EmpPlotFreq-1}
\label{fig:EmpPlotFreq-1}
\end{figure}

To assess the process focus of the broader empirical works, we again
assessed when in the migration process the data was collected and we
additionally assessed the type analysis done by the authors. We found
that 512 studies (97.71\%) collected data after the arrival of the
migrant in the new society. Two studies targeted potential migrants and
10 studies collected data prior to and following the migration event.
Moreover, only 25 studies included longitudinal data analyses of
psychological acculturation (4.79\%). This observation again underscores
the arguments that the acculturation literature has thus far failed to
provide data that meaningfully captures migration as a process
\citep[e.g.,][]{Brown2011, Ward2019}.

To further assess the comparative utility of the experience framework,
we then assessed differences of experience aspects between academic
fields. For the full results, including differences in the methods, and
publication types as well as contextual differences in terms of sampling
procedures, situational domains, analyses, and cultural contexts see
Supplemental Material B.

We first assessed the references to affect, behavior, cognition, and
desires separately, for each of the disciplines. We find that for all
fields desires (12.5-28.69\% of all measures in the field) and emotions
(35.56-62.3\%) are the least frequently measured elements and medical
journals measure them the least frequently (in proportional terms).
Looking at the common cognitive and behavioral elements the proportions
diverge between the fields. While the multidisciplinary field measured
behaviors (76.47\%) and cognition (82.35\%) almost equally often, in the
medical and general social science journals behaviors were measured
considerably more often than cognitions (\(Behavior_{SoSci}\) = 86.67\%
\textgreater{} \(Cognition_{SoSci}\) = 68.89\%; \(Behavior_{Med}\) =
89.58\% \textgreater{} \(Cognition_{Med}\) = 69.44\%). Inversely, in the
psychological journals cognitions (90.98\%) were measured more often
than behaviors (68.03\%; also see Figure \ref{fig:FieldPlotFreq} A and
B).

When looking at differences in how many different experience aspects
were measured together and patterns within these aspect-combinations,
differences between the fields become increasingly evident (also see
Figure \ref{fig:FieldPlotFreq} A and C). While `affect, behavior, and
cognition' and `behavior, and cognition' measures are common
combinations across all fields, fewer experience aspects and less
variation were considered in the medical and social science fields.
There were statistically significant mean differences between the fields
in terms of how many experience aspects were considered (parametric:
\textit{F}(3, 409) = 5.02, \textit{p} = 0.002, non-parametric:
\textit{Kruskal-Wallis} \(\chi^{2}\) = 15.01, \textit{df} = 3,
\textit{p} = 0.002, \(\eta_{p}^{2}\) = 0.04, 95\%CI{[}0.01, 1{]}).
Looking at the mean differences in more detail, empirical works
published in psychological journals had significantly higher average
aspect counts (\textit{M} = 2.5, \textit{SD} = 0.83) than the medical
(\textit{M} = 2.1, \textit{SD} = 0.86) and the general social science
journals (\textit{M} = 2.04, \textit{SD} = 0.73; also see Figure
\ref{fig:FieldPlotComplexityAverage}). The broader patterns described
here thus point to heterogeneity between fields and show that different
fields diverge in the number and types of acculturation aspects they
tend to consider.

\begin{figure}[h]
\centering
\caption{Scale Complexity and their proportional occurences per field.}
\includegraphics[width=\textwidth]{Figures/FieldPlotComplexityAverage-1}
\label{fig:FieldPlotComplexityAverage}
\end{figure}

\begin{figure}[h]
\centering
\caption{Combinations of measured experience element and their frequencies per field.}
\includegraphics[width=\textwidth]{Figures/FieldPlotFreq-1}
\label{fig:FieldPlotFreq}
\end{figure}

\begin{figure}[h]
\centering
\caption{Literature Levels: (A) Bar graph of the experience aspect frequencies for theoretical, methodological, and broader empirical literature. (B) Bar graph of the number of experience aspects used for theoretical, methodological, and broader empirical literature. (C) Average number of additional aspects included when the aspect was considered for theoretical, methodological, and broader empirical literature [Mean ± 95\%CI].}
\includegraphics[width=\textwidth]{Figures/LiteratureComparison-1}
\caption*{Note that in (C) within each literature body the aspects are not mutually exclusive (and thus not independent) because scales can include multiple experience aspects.}
\label{fig:LiteratureComparison}
\end{figure}
