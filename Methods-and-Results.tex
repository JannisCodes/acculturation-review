\subsection{Methodological Literature}

Based on the systematic review and its coding, the first data set we
assess is a database of scale validations. We bring together the scales
suggested in previous reviews as well as validation studies we
identified in our own review. Throughout our literature review we found
five major works that reviewed the measurement of acculturation
\citep{Celenk2011, Maestas2000, Matsudaira2006, Wallace2010, Zane2004}.
After the removal of duplicate scales, we added any scale validation
that was present in our own systematic review but not included in the
previous reviews. For each measure we extracted the full item list as
well as the item scoring prior to coding. A comprehensive and
interactive database of the scales, with reference- and publication
information, as well as our experience elements and -context coding is
available in our online supplementary information as well as on our open
science repository (OSF and/or github citation here).

\subsubsection{Methods}

Taken together these five reviews collected a total of 197 scales, of
which 75 were duplicates. From our own review we added 25 additional
validation studies. After removing duplicates this meant that we
considered a total of 122 unique scales for our coding. Of these scales
we ultimately had to exclude 41, because they were either not accessible
or did not fit the the topic of our review (see Table
\ref{tab:ScalesExclusion}). The scales had an average of \hl{X.XX} items
and \hl{X.XX} sub-scales. Most items were rated on a five-point
(\hl{XX.XX}\%) or four-point likert-type scale (\hl{XX.XX}\%), with only
\hl{X} scales including categorical ratings. About a fifth of scales
(20.4918\%) included majority group members in their validation studies.
The earliest included validation was from 1972 with a majority of scales
being validated around the turn of the 21\textsuperscript{st} century
and the latest included validation study in 2018.

\begin{table}

\caption{(\#tab:ScalesExclusion)Reasons for Exclusion}
\begin{tabular}[t]{lc}
\toprule
Exclusion Reason & Frequency\\
\midrule
items not accessible & 15\\
article not accessible & 4\\
chapter not accessible & 1\\
not acculturation & 1\\
not measured & 1\\
thesis not accessible & 1\\
\bottomrule
\end{tabular}
\end{table}


\subsubsection{Results}

For the literature on scale validations, we assessed both the role of
experience elements in the measures as well as contextual differences.

\paragraph{Experience}

With our main aim of examining the experience structure within the
scales, we examined whether scales included a specific experience
elements but also examined the used elements in their complex
combinations. In terms of general inclusion of elements, most studies
included a measure of cognition (89.66\%) and behavior (82.76\%),
whereas only roughly half the studies included a measure of affect
(55.17\%) and only a fourth of the scales included a measure of motives
(28.74\%). However, only a minority of scales included only a single
dimension. There were only 5 scales that exclusively relied on
cognitions (5.75\%) and 4 scales that measured only behaviors (4.6\%).
Yet, inversely, there were also only 13 scales that measured all four
dimensions (14.94\%). Most studies measured two (37.93\%) or three
(36.78\%) dimensions. A majority of scales either measured behavioral
and cognitive elements (26.44\%) or behavioral, cognitive, and affective
elements (26.44\%; also see Figure \ref{fig:ElementsScales} and Table
\ref{tab:ElementCooccurances}). Looking at the number of elements
measured together we also see substantial differences in what kind of
scales include a certain element. Scales that included cognitions
measured an average of 2.67 elements, scales measuring behavior, on
average, measured a 2.71, while scales that included affect measures had
a complexity average of 3.1 and scales measuring motivation even
measured an average of 3.4 scales. Thus, most scales measure multiple
dimensions, yet they focus on easily accessible dimensions (i.e.,
behavior and cognition), less of what is considered `less accessible' or
`subjective' (i.e., affect and desires). This is also visible in the
circumstance that there were no scales that exclusively measured
motivational or emotional adaptation (while this was the case for both
cognitions and behaviors). And if emotional or motivational aspects were
measured they were on average measured in scales that were already more
complex (i.e., included more experience elements).

\begin{figure}[h]
\centering
\caption{Bar graph of the experience element combinations.}
\includegraphics[width=\textwidth]{Figures/ABCDFreq-1}
\label{fig:ElementsScales}
\end{figure}

\begin{table}

\caption{\label{tab:ElementCooccurrencess}Element Co-occurrence Matrix}
\begin{tabular}[t]{lcccc}
\toprule
  & Affect & Behavior & Cognition & Desire\\
\midrule
Affect & 48 & 40 & 43 & 18\\
Behavior & 40 & 72 & 64 & 19\\
Cognition & 43 & 64 & 78 & 23\\
Desire & 18 & 19 & 23 & 25\\
\bottomrule
\end{tabular}
\end{table}


\paragraph{Context}

To gain a general understanding of contextual factors within the
validated studies, we also assessed cross-study patterns of cultural,
individual, situational, and process-related focus points.

\subparagraph{Country}

To assess the cultural contexts for which scales were validated we
assessed the migrants' countries of settlement as well as the countries
of origin. We found that most scales investigated a single host country
(\textit{N} = 81) and most investigated one country of origin
(\textit{N} = 68). There was 1 scale that was validated in two countries
and 1 that was validated in three countries. Additionally, there was a
single scale that was validated in a larger multi-national survey
context (i.e., with multiple host and origin countries). There was also
one study with two scales that were validated with the origin culture as
the starting point (i.e., single origin country, multiple host
countries). Looking at the country patterns, we found that an
overwhelming number of scales were validated within a U.S. American
settlement context (\textit{N} = 61). But also the remaining receiving
societies were mostly `western' countries (e.g., Canada, The
Netherlands, The United Kingdom, Israel, Australia) with only individual
scales for Taiwanese, Nepalese, or Russian settlement contexts. For the
migrant origin societies there was slightly more variation. There were a
few migrant groups that were investigated specifically (e.g., Mexico:
14, China:7, South Korea: 4), however most validation studies targeted
broader categories of migrants (any migrants: 11, Asian: 5, Hispanic: 9,
LatinX: 5). This also made it difficult to identify patterns of cultural
combinations investigated (apart from Mexican and LatinX migrants in the
United States).

\subparagraph{Sample}

To assess the role different groups of individuals targeted in the scale
validations, we coded the types of samples recruited for the validation
studies. A majority of studies sampled any consenting adult from the
migrant group of interest (\textit{N} = 50). As seems common in academic
research, a larger portion of the validated scales relied on young
migrants or students (\textit{N} = 29). Interestingly, only small
minority of validated scales targeted more vulnerable groups, such as
clinical samples (\textit{N} = 2) or refugees (\textit{N} = 2) --
despite a considerable focus on these groups within the broader
literature.

\subparagraph{Domains}

To assess the situational focus within the validated scales, we assessed
the number of domains within each scale as well as more common domains
across the scales. A relatively large number of scales asked about the
current state of the migrant in general manner without mentioning any
context or life domain (\textit{N} = 24; e.g., `'In general, in what
language do you read and speak?'`). The remaining scales referred to an
average of 3.49 dimensions (\textit{SD} = 3.42, range: 1 -- 21). A total
of 179 unique domains was measured across the 87 scales. The domains of
'language` (\hl{XX}\%), 'food' (\hl{XX}\%), `interactions' (\hl{XX}\%),
`family' (\hl{XX}\%) and `values' (\hl{XX}\%) were focused on most often
(see Online Supplementary Materials \hl{X}, Figure \hl{X}). Thus, while
there was large variation between the scales in the number, and
diversity of domains, the most frequently mentioned domains were in line
with the life domains proposed in the literature
\citep[e.g.,][]{Arends-Toth2007}.

\vspace{1em}
\todo[inline]{Should be recoded to test our proposed domains. Also, re-check `general' code}

END SECTION
