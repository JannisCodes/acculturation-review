% define document type (i.e., template. Here: A4 APA manuscript with 12pt font)
\documentclass[man, 12pt, a4paper]{apa7}

% add packages
\usepackage[american]{babel}
\usepackage[utf8]{inputenc}
\usepackage{csquotes}
\usepackage{hyperref}
\usepackage[style=apa, sortcites=true, sorting=nyt, backend=biber, natbib=true, uniquename=false, uniquelist=false, useprefix=true]{biblatex}
\usepackage{authblk}
\usepackage{graphicx}
\usepackage{setspace,caption}
\usepackage{subcaption}
\usepackage{enumitem}
\usepackage{lipsum}
\usepackage{soul}
\usepackage{xcolor}
\usepackage{fourier}
\usepackage{stackengine}
\usepackage{scalerel}
\usepackage{fontawesome}
\usepackage[normalem]{ulem}
\usepackage{longtable}
\usepackage{amsmath}
\usepackage{afterpage}
\usepackage{float}
\usepackage{titling}
\usepackage{censor}
\usepackage{tcolorbox}

% formatting links in the PDF file
\hypersetup{
pdfpagemode={UseOutlines},
bookmarksopen=true,
bookmarksopenlevel=0,
hypertexnames=false,
colorlinks   = true, %Colours links instead of ugly boxes
urlcolor     = blue, %Colour for external hyperlinks
linkcolor    = blue, %Colour of internal links
citecolor   = cyan, %Colour of citations
pdfstartview={FitV},
unicode,
breaklinks=true,
}

% language settings
\DeclareLanguageMapping{american}{american-apa}

% add reference library file
\addbibresource{references.bib}

% Title and header
\title{Supplemental Information D: Acculturation Scale Directory}
\shorttitle{SI D: Acculturation Scale Directory}
%\author{Jannis Kreienkamp, Laura F. Bringmann, Raili F. Engler, Peter de Jonge, Kai Epstude}
\author{[authors masked for peer review]}

% set indentation size
\setlength\parindent{1.27cm}

% adapt table and figure labels
\setcounter{equation}{0}
\setcounter{figure}{0}
\setcounter{table}{0}
\setcounter{page}{1}
\makeatletter
\renewcommand{\theequation}{S\arabic{equation}}
\renewcommand{\thefigure}{S\arabic{figure}}
\renewcommand{\thetable}{S\arabic{table}}

% Start of the main document:
\begin{document}

% add title information (incl. title page and abstract)
\begin{titlepage}
	{\noindent\Large Supplementary Information for \par}
	\vspace{0.5cm}
	{\noindent\Large The Migration Experience: A Conceptual Framework and Systematic Review of Psychological Acculturation\par}
	\vspace{1.5cm}
	{\noindent\LARGE\bfseries \thetitle \par}
	\vspace{2cm}
	{\noindent\Large\itshape \theauthor \par}
	\vfill
	%\noindent Corresponding Author: Jannis Kreienkamp\par
	%\noindent E-mail: j.kreienkamp@rug.nl\par
	\noindent Corresponding Author: [masked for peer review]\par
	\noindent E-mail: [masked for peer review]\par
	\vfill

    % Bottom of the page
	{\noindent Last updated: \today\par}
\end{titlepage}

% add title again on page 1 (after title page)
\begin{center}
   \textbf{\thetitle} 
\end{center}

This supplementary information introduces the acculturation scale directory. As part of the systematic review we collected and coded a wide range of methodological manuscripts on psychological acculturation. From the literature we extracted 256 scales that have been used to assess acculturation. Of these scales we were able to include and code 233 acculturation scales. Given that this undertaking, to the best of our knowledge, brought together the largest collection of acculturation scales to date, we decided to make the scales and their attributes accessible to the readership. To this aim, we created an interactive scale directory, which is available here:

\vspace{.5cm}
\begin{tcolorbox}
    \vspace{0.2cm} \centering 
    \href{https://acculturation-review.shinyapps.io/scale-directory/}{https://acculturation-review.shinyapps.io/scale-directory/}
    \vspace{0.2cm} 
\end{tcolorbox}

\section{Features}
This scale directory has three main functions, as it aims to (1) aid measurement selection, (2) measurement accessibility, and (3) exploration of the review results.

The most practical function of this application is to aid researchers and practitioners in the selection of acculturation measurements. The study of acculturation has produced an immense number of acculturation scales, and making a choice between these different tools can be difficult. Not only is it difficult to gain an overview of the number of scales used within the literature, also the diversity in style and content can be overwhelming. We hope that the filter options we provide in our application can offer a first theory-based and intuitive entry into the plethora for acculturation scales. It should be noted that this directory is not meant to replace a full methodological review and does only present a small amount of information on the scales.

We also hope to make the scales easily accessible to the users of the application. We do so by showcasing all (publicly) available scale items by clicking the eye icon in the 'View' column. We, additionally, list the full reference to the scale in the References tab (linked via a 'Reference' column in the scale directory). 

And finally, as part of the framework development and systematic review, we have arrived at a number of conclusions about the methodological literature on acculturation. We hope that readers can use this directory in conjunction with the main article and explore the results themselves. The data table and the appended filter allow readers an interactive access to the data and users might gain an intuitive understanding of the current state of the literature.


\section{Interface}
Users of the application arrive at the scale database itself, where they have an interactive access to the scales themselves. There are two additional windows (i.e., tabs) available to supplement bibliographic information of the scales (i.e., references tab) and an introduction to the application and the broader review (i.e., about tab). Yet the core element of the application remains the scale directory itself, which consists of three main interface elements, (1) an interactive data table of the scales, (2) a filter section, and (3) short information box.

The visually largest space is taken up be the scale data table, which allows direct access to the scale directory. The table shows all scales that fit the current filters and lists a number of key information about the scale. Next to the name of the scale, the APA short reference, the number or items, and number of life domains, the overview also indicates whether the scale included any of the affect, behavior, cognition, and/or desire aspects. Users can interact with this data table by sorting the scales based on their column values and the first column is a click-able area, which gives access to additional information about the scale. Wherever (publicly) available, we list the exact items, the response options, the life domains considered, as well as some information on the validation sample.

The filter section contains the main mechanisms for interacting with the scale directory. We currently offer three main filters to identify scales that fit the users' needs and more generally allow for exploration of the methodological literature.
\begin{itemize}
\item The `Experience Aspect Filter' allows to filter the inclusion of the affect, behavior, cognition, and/or desire aspects. If this filter is disabled any combination of experience aspects will be displayed. Once enabled, the data table will display all scales that fit the user's experience aspect focus.
\item The `Number of Items Filter' allows to filter the acculturation scales by the number of items within the scale. Users can use a horizontal slider to select the minimum and maximum number of items the scale is allowed to have.
\item The `Number of Domains Filter' allows to filter the scales by the number of life domains (i.e., situational contexts) assessed within the scale. Users can use a second horizontal slider to select the minimum and maximum number of domains that should be assessed within the scales.
\end{itemize}

The final, information section offers a top-level overview of the current scale selection. The current version shows the number scales that fit the current filter choices, the average number of items of the selected scales, the total number of items of all selected scales, as well as a short general introduction to the directory.

\printbibliography

\end{document}