% define document type (i.e., template. Here: A4 APA manuscript with 12pt font)
\documentclass[man, 12pt, a4paper]{apa7}

% add packages
\usepackage[american]{babel}
\usepackage[utf8]{inputenc}
\usepackage{csquotes}
\usepackage{hyperref}
\usepackage[style=apa, sortcites=true, sorting=nyt, backend=biber, natbib=true, uniquename=false, uniquelist=false, useprefix=true]{biblatex}
\usepackage{authblk}
\usepackage{graphicx}
\usepackage{setspace,caption}
\usepackage{subcaption}
\usepackage{enumitem}
\usepackage{lipsum}
\usepackage{soul}
\usepackage{xcolor}
\usepackage{fourier}
\usepackage{stackengine}
\usepackage{scalerel}
\usepackage{fontawesome}
\usepackage[normalem]{ulem}
\usepackage{longtable}
\usepackage{amsmath}
\usepackage{afterpage}
\usepackage{float}
\usepackage{titling}
\usepackage{censor}

% formatting links in the PDF file
\hypersetup{
pdfpagemode={UseOutlines},
bookmarksopen=true,
bookmarksopenlevel=0,
hypertexnames=false,
colorlinks   = true, %Colours links instead of ugly boxes
urlcolor     = blue, %Colour for external hyperlinks
linkcolor    = blue, %Colour of internal links
citecolor   = cyan, %Colour of citations
pdfstartview={FitV},
unicode,
breaklinks=true,
}

% language settings
\DeclareLanguageMapping{american}{american-apa}

% add reference library file
\addbibresource{references.bib}

% Title and header
\title{Appendix A: Search Strategy}
\shorttitle{Appendix A: Search Strategy}
%\author{Jannis Kreienkamp, Laura F. Bringmann, Raili F. Engler, Peter de Jonge, Kai Epstude}
\author{[authors masked for peer review]}

% set indentation size
\setlength\parindent{1.27cm}

% adapt table and figure labels
\setcounter{equation}{0}
\setcounter{figure}{0}
\setcounter{table}{0}
\setcounter{page}{1}
\makeatletter
\renewcommand{\theequation}{S\arabic{equation}}
\renewcommand{\thefigure}{S\arabic{figure}}
\renewcommand{\thetable}{S\arabic{table}}

% Start of the main document:
\begin{document}

% add title information (incl. title page and abstract)
\begin{titlepage}
	{\noindent\Large Appendix for \par}
	\vspace{0.5cm}
	{\noindent\Large The Migration Experience: A Conceptual Framework and Systematic Review of Psychological Acculturation\par}
	\vspace{1.5cm}
	{\noindent\LARGE\bfseries \thetitle \par}
	\vspace{2cm}
	{\noindent\Large\itshape \theauthor \par}
	\vfill
	%\noindent Corresponding Author: Jannis Kreienkamp\par
	%\noindent E-mail: j.kreienkamp@rug.nl\par
	\noindent Corresponding Author: [masked for peer review]\par
	\noindent E-mail: [masked for peer review]\par
	\vfill

    % Bottom of the page
	{\noindent Last updated: \today\par}
\end{titlepage}

% add title again on page 1 (after title page)
%\begin{center}
%   \textbf{\thetitle} 
%\end{center}

\appendix

\section{Search Strategy}
\label{app:AppendixSearchStrategy}

To assess the past empirical and theoretical literature on psychological acculturation, we performed a systematic review. We first read seminal and review works within the field \citep[including,][]{Ward2019, Berry1997b, Berry2003, Szapocznik1978, Sam2006a, Rudmin2003a}. Based on our reading of the literature, we designed a comprehensive literature search strategy in an iterative fashion. 

For the empirical work on acculturation we performed a literature search on March 4\textsuperscript{th}, 2020 and February 14\textsuperscript{th}, 2021, within the ``APA PsycINFO'' bibliographic databases using the EBSCO\textit{host} provider. The databases also included the PsycARTICLES, PsycBOOKS, and PsycCRITIQUES databases as well ProQuest Dissertations with psychological relevance. The second literature search included alternate terms used less frequently to describe what we mean with psychological acculturation, including "transculturation" and "cultural transition". Additionally, the second search removed limiter terms that could have exclude interdisciplinary investigations and focused on human participants.

For the theoretical literature performed an additional, more specific, search of the same databases as well as the Web of Science Core Collection using the Clarivate Analytics provider on March 3\textsuperscript{rd}, 2021.

In designing our search strategy we used an adapted version of the `SPIDER' research tool \citep[e.g.,][]{Cooke2012}. We utilized the \textit{Evaluation} element mainly to exclude articles that were not relevant to the search. The exact search terms used are listed in Table \ref{tab:SearchStrategiesTab} below.

\newpage

\begin{table}
\caption{Search Strategy Cultural Adaptation Review}
\label{tab:SearchStrategiesTab} 
\begin{tabular}{ll}
\hline
Element & Search Terms \\ 
\hline \\ [-0.5em]

% Sample
Sample & 
    (Immigration OR migration OR migrant OR immigration OR refugee) \\ 
    \\ [-0.25em]
    
% Phenomenon of Interest
\begin{tabular}[t]{@{}l@{}}Phenomenon \\of Interest\end{tabular} & 
    \begin{tabular}[t]{@{}l@{}}(acculturation OR enculturation OR transculturation OR \\
    assimilation OR "social integration" OR "cultural adaptation" OR \\
    "cultural adjustment" OR "cultural transition")\end{tabular}  \\ 
    \\ [-0.25em]
    
% Design
Design & 
    \begin{tabular}[t]{@{}l@{}}("measurement tool" OR scale OR instrument OR \\
    questionnaire OR survey OR definition OR inventory)\end{tabular}  \\ 
    \\ [-0.25em]
    
% Evaluation
Evaluation & 
    \begin{tabular}[t]{@{}l@{}}NOT (parent* OR college OR resilience OR treatment OR \\
    intervention OR therapy)\end{tabular}  \\ 
    \\ [-0.25em]
    
% Research Type
Research type * & 
    (quantitative OR qualitative OR "mixed method") \\ [0.75em] 
    \hline

% Note
\multicolumn{2}{l}{* This element was ultimately dropped because it was too sensitive in PsycInfo.}
\end{tabular}
\end{table}

\newpage

\subsection{References}

\printbibliography[heading=none]

\end{document}